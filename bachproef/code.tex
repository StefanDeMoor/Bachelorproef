%%=============================================================================
%% Code
%%=============================================================================
\lstset{
    breaklines=true,
    numbers=left,
    frame=single,
    backgroundcolor=\color{gray!10}
}

\section{Loginpagina}

In dit onderdeel wordt de loginpagina van de applicatie besproken. De implementatie is opgedeeld in drie delen: de gebruikersinterface (XAML), de logica in de ViewModel (C\#) en de clientservice voor authenticatieverzoeken (C\#). Deze structuur volgt het MVVM-patroon.

\subsection{XAML – Gebruikersinterface}

\begin{lstlisting}[language=XML, caption=Gebruikersinterface van de loginpagina, label=lst:login-xaml]
    <?xml version="1.0" encoding="utf-8" ?>
    <ContentPage xmlns="http://schemas.microsoft.com/dotnet/2021/maui"
    xmlns:x="http://schemas.microsoft.com/winfx/2009/xaml"
    xmlns:viewmodel="clr-namespace:ATS.ViewModels"
    x:DataType="viewmodel:LoginPageViewModel"
    x:Class="ATS.Views.LoginPage"
    Title="Login"
    Shell.NavBarIsVisible="False">
    
    <Grid>
    <Grid.RowDefinitions>
    <RowDefinition Height="220" />
    <RowDefinition Height="*" />
    <RowDefinition Height="50" />
    </Grid.RowDefinitions>
    
    <Path Data="m0.30001,0l449,0l0,128.48327c-122.5,64.30167 -332.5,65.89076 -449,7.2429c0,-45.25313 0,-90.47304 0,-135.72617z" Fill="CornflowerBlue" />
    
    <StackLayout>
    <Border BackgroundColor="LightGray" HeightRequest="60" WidthRequest="60" Margin="0, 100, 0, 0" HorizontalOptions="Center">
    <Border.StrokeShape>
    <RoundRectangle CornerRadius="30"/>
    </Border.StrokeShape>
    <Image Source="user" Aspect="AspectFill" />
    </Border>
    </StackLayout>
    
    <Grid Grid.Row="1" Margin="20, 20, 20, 0" RowSpacing="5">
    <Grid.RowDefinitions>
    <RowDefinition Height="*" />
    <RowDefinition Height="50" />
    <RowDefinition Height="50" />
    <RowDefinition Height="Auto" />
    <RowDefinition Height="40" />
    <RowDefinition Height="40" />
    <RowDefinition Height="*" />
    </Grid.RowDefinitions>
    
    <Label Grid.Row="1" Text="Welcome!" HorizontalOptions="Center" FontSize="Title" FontAttributes="Bold"/>
    <Label Grid.Row="2" Text="Sign in to continue" HorizontalOptions="Center" FontSize="Subtitle"/>
    
    <StackLayout Grid.Row="3">
    <Border Stroke="CornflowerBlue">
    <Border.StrokeShape><RoundRectangle CornerRadius="30"/></Border.StrokeShape>
    <Grid ColumnDefinitions="Auto,*">
    <Border BackgroundColor="LightGray" HeightRequest="40" Margin="5">
    <Border.StrokeShape><RoundRectangle CornerRadius="30"/></Border.StrokeShape>
    <Image Source="user" WidthRequest="40" HeightRequest="40"/>
    </Border>
    <Entry Text="{Binding LoginModel.Email}" Placeholder="Username" FontAttributes="Bold"/>
    </Grid>
    </Border>
    
    <Border Stroke="CornflowerBlue" Margin="0, 15, 0, 0">
    <Border.StrokeShape><RoundRectangle CornerRadius="30"/></Border.StrokeShape>
    <Grid ColumnDefinitions="Auto,*">
    <Border BackgroundColor="LightGray" HeightRequest="40" Margin="5">
    <Border.StrokeShape><RoundRectangle CornerRadius="30"/></Border.StrokeShape>
    <Image Source="lock" WidthRequest="40" HeightRequest="40"/>
    </Border>
    <Entry Text="{Binding LoginModel.Password}" Placeholder="Password" FontAttributes="Bold"/>
    </Grid>
    </Border>
    
    <Label Text="{Binding ErrorMessage}" TextColor="Red" IsVisible="{Binding IsErrorVisible}" />
    
    <Grid ColumnDefinitions="Auto,*,Auto" Margin="0, 10, 0, 0">
    <CheckBox Grid.Column="0" IsChecked="False"/>
    <Label Grid.Column="1" Text="Remember me"/>
    <Label Grid.Column="2" Text="Forgot Password?" TextColor="CornflowerBlue" FontAttributes="Bold" HorizontalOptions="End"/>
    </Grid>
    
    <Button Text="SIGN IN" BackgroundColor="CornflowerBlue" TextColor="White" FontAttributes="Bold" CornerRadius="30" WidthRequest="200" Margin="0, 15, 0, 0" Command="{Binding LoginCommand}"/>
    </StackLayout>
    </Grid>
    </Grid>
    </ContentPage>
\end{lstlisting}

\subsection{ViewModel – Logica en Binding}

\begin{lstlisting}[language=C, caption=ViewModel voor loginpagina, label=lst:login-viewmodel]
    using CommunityToolkit.Mvvm.ComponentModel;
    using CommunityToolkit.Mvvm.Input;
    using ATS.Models;
    using ATS.Views;
    using ATS.Services;
    using ATS.Services.Users;
    
    namespace ATS.ViewModels
    {
        public partial class LoginPageViewModel : ObservableObject
        {
            public required RegisterModel RegisterModel { get; set; }
            public required LoginModel LoginModel { get; set; }
            
            private readonly ClientService clientService;
            private readonly UserService userService;
            
            public LoginPageViewModel(ClientService clientService, UserService userService)
            {
                this.clientService = clientService;
                this.userService = userService;
                RegisterModel = new RegisterModel();
                LoginModel = new LoginModel();
                IsAuthenticated = false;
            }
            
            [RelayCommand]
            private async Task Register()
            {
                await clientService.Register(RegisterModel);
            }
            
            [RelayCommand]
            private async Task Login()
            {
                bool loginSuccess = await clientService.Login(LoginModel);
                
                if (loginSuccess)
                {
                    var user = await userService.GetCurrentUserAsync();
                    if (user != null && !string.IsNullOrEmpty(user.UserName))
                    {
                        await userService.SetCurrentUserAsync(user);
                        IsAuthenticated = true;
                        UserName = user.UserName;
                        await Shell.Current.GoToAsync($"//{nameof(HomePage)}");
                    }
                    else
                    {
                        ShowError("Invalid login credentials.");
                    }
                }
                else
                {
                    ShowError("Login failed. Please try again.");
                }
            }
            
            private void ShowError(string message)
            {
                IsErrorVisible = true;
                ErrorMessage = message;
            }
            
            [RelayCommand]
            private async Task Logout()
            {
                userService.Logout();
                IsAuthenticated = false;
                UserName = string.Empty;
                await Shell.Current.GoToAsync("..");
            }
            
            public string? UserName { get; set; }
            public string? ErrorMessage { get; set; }
            public bool IsErrorVisible { get; set; }
            public bool IsAuthenticated { get; set; }
        }
    }
\end{lstlisting}

\subsection{ClientService – Verwerken van inlogverzoeken}

\begin{lstlisting}[language=C, caption=Authenticatie via de ClientService, label=lst:login-clientservice]
    using ATS.Models;
    using System.Net.Http.Json;
    using System.Text.Json;
    using Shared.DTOs;
    using Plugin.Firebase.CloudMessaging;
    using ATS.Services.Users;
    
    namespace ATS.Services
    {
        public class ClientService
        {
            private readonly IHttpClientFactory _httpClientFactory;
            private readonly UserService _userService;
            
            public ClientService(IHttpClientFactory httpClientFactory, UserService userService)
            {
                _httpClientFactory = httpClientFactory;
                _userService = userService;
            }
            
            public async Task Register(RegisterModel model)
            {
                var httpClient = _httpClientFactory.CreateClient("custom-httpclient");
                
                var requestDto = new RegisterRequestDto
                {
                    Email = model.Email,
                    Password = model.Password
                };
                
                var result = await httpClient.PostAsJsonAsync("/api/auth/register", requestDto);
                result.EnsureSuccessStatusCode(); 
            }
            
            public async Task<bool> Login(LoginModel model)
            {
                var httpClient = _httpClientFactory.CreateClient("custom-httpclient");
                var requestDto = new LoginRequestDto
                {
                    Email = model.Email,
                    Password = model.Password
                };
                
                var result = await httpClient.PostAsJsonAsync("/api/auth/login", requestDto);
                
                if (result.IsSuccessStatusCode)
                {
                    var response = await result.Content.ReadFromJsonAsync<LoginResponseDto>();
                    
                    if (response != null && !string.IsNullOrEmpty(response.UserName))
                    {
                        var serialized = JsonSerializer.Serialize(response);
                        await SecureStorage.Default.SetAsync("Authentication", serialized);
                        
                        var token = await CrossFirebaseCloudMessaging.Current.GetTokenAsync();
                        if (!string.IsNullOrEmpty(token))
                        {
                            await _userService.AddFcmTokenAsync(token);
                        }
                        
                        return true;
                    }
                }
                
                return false;
            }
        }
    }
\end{lstlisting}

\section{Pushnotificatie}

In deze sectie wordt de implementatie besproken van pushnotificaties binnen de applicatie. Deze is opgebouwd uit drie hoofdcomponenten: de ViewModel waarin notificaties verzonden worden, de Android-specifieke configuratie in de hoofdactiviteit, en de algemene configuratie van Firebase in het opstartbestand van de applicatie.

\subsection{HomePageViewModel – Verwerken en verzenden van pushnotificaties}

\begin{lstlisting}[language=C, caption=Versturen van pushnotificaties in HomePageViewModel, label=lst:pushnotification-viewmodel]
    using ATS.Views;
    using CommunityToolkit.Mvvm.ComponentModel;
    using CommunityToolkit.Mvvm.Input;
    using ATS.Services.Users;
    using Plugin.Firebase.CloudMessaging;
    using FirebaseAdmin;
    using FirebaseAdmin.Messaging;
    using Google.Apis.Auth.OAuth2;
    
    namespace ATS.ViewModels
    {
        public partial class HomePageViewModel : ObservableObject
        {
            private readonly UserService _userService;
            private string? _userRole;
            private string? _fcmToken;
            
            public HomePageViewModel(UserService userService)
            {
                _userService = userService;
                Task.Run(async () => await InitializeUserRole());
            }
            
            public bool IsButtonVisible => !string.IsNullOrEmpty(_userRole) && _userRole != "Guest";
            
            [RelayCommand]
            private async Task GoToData()
            {
                await Shell.Current.GoToAsync($"//{nameof(DataPage)}");
            }
            
            [RelayCommand]
            private async Task SendPushNotification()
            {
                var app = FirebaseApp.Create(new AppOptions
                {
                    Credential = await GetCredential()
                });
                
                FirebaseMessaging messaging = FirebaseMessaging.GetMessaging(app);
                var tokens = await _userService.GetFcmTokensAsync();
                
                foreach (var token in tokens)
                {
                    var message = new Message()
                    {
                        Token = token,
                        Notification = new Notification { Title = "Hello world!", Body = "It's a message for Android with MAUI" },
                        Data = new Dictionary<string, string> { { "greating", "hello" } },
                        Android = new AndroidConfig { Priority = Priority.Normal },
                        Apns = new ApnsConfig { Headers = new Dictionary<string, string> { { "apns-priority", "5" } } }
                    };
                    
                    try
                    {
                        var response = await messaging.SendAsync(message);
                        Console.WriteLine($"Sent to {token}: {response}");
                    }
                    catch (Exception ex)
                    {
                        Console.WriteLine($"Error sending to {token}: {ex.Message}");
                    }
                }
                
                await Shell.Current.DisplayAlert("Push", $"Sent to {tokens.Count} device(s)", "OK");
            }
            
            private async Task<GoogleCredential> GetCredential()
            {
                var path = await FileSystem.OpenAppPackageFileAsync("firebase-adminsdk.json");
                return GoogleCredential.FromStream(path);
            }
            
            private async Task InitializeUserRole()
            {
                _userRole = await _userService.GetUserRoleAsync();
                OnPropertyChanged(nameof(IsButtonVisible));
            }
        }
    }
\end{lstlisting}

\subsection{Android – Aanmaken van notificatiekanaal}

\begin{lstlisting}[language=C, caption=Android configuratie in MainActivity, label=lst:pushnotification-android]
    using Android.App;
    using Android.Content;
    using Android.Content.PM;
    using Android.OS;
    using Plugin.Firebase.CloudMessaging;
    
    namespace ATS
    {
        [Activity(Theme = "@style/Maui.SplashTheme", MainLauncher = true, LaunchMode = LaunchMode.SingleTop,
        ConfigurationChanges = ConfigChanges.ScreenSize | ConfigChanges.Orientation |
        ConfigChanges.UiMode | ConfigChanges.ScreenLayout | ConfigChanges.SmallestScreenSize |
        ConfigChanges.Density)]
        public class MainActivity : MauiAppCompatActivity
        {
            protected override void OnCreate(Bundle? savedInstanceState)
            {
                base.OnCreate(savedInstanceState);
                Firebase.FirebaseApp.InitializeApp(this);
                HandleIntent(Intent);
                CreateNotificationChannelIfNeeded();
            }
            
            protected override void OnNewIntent(Intent? intent)
            {
                base.OnNewIntent(intent);
                HandleIntent(intent);
            }
            
            private static void HandleIntent(Intent? intent)
            {
                FirebaseCloudMessagingImplementation.OnNewIntent(intent);
            }
            
            private void CreateNotificationChannelIfNeeded()
            {
                if (Build.VERSION.SdkInt >= BuildVersionCodes.O)
                {
                    CreateNotificationChannel();
                }
            }
            
            private void CreateNotificationChannel()
            {
                var channelId = $"{PackageName}.general";
                var notificationManager = (NotificationManager)GetSystemService(NotificationService);
                var channel = new NotificationChannel(channelId, "General", NotificationImportance.Default);
                notificationManager.CreateNotificationChannel(channel);
                FirebaseCloudMessagingImplementation.ChannelId = channelId;
            }
        }
    }
\end{lstlisting}

\subsection{MauiProgram.cs – Firebase configuratie en dependency injection}

\begin{lstlisting}[language=C, caption=Firebase initialisatie en dependency injection, label=lst:pushnotification-mauiprogram]
    using ATS.Services;
    using ATS.Services.Users;
    using ATS.ViewModels;
    using ATS.Views;
    using Microsoft.Extensions.Logging;
    using Microsoft.Maui.LifecycleEvents;
    using Plugin.Firebase.CloudMessaging;
    
    #if IOS
    using Plugin.Firebase.Core.Platforms.iOS;
    #elif ANDROID
    using Plugin.Firebase.Core.Platforms.Android;
    #endif
    
    namespace ATS
    {
        public static class MauiProgram
        {
            public static MauiApp CreateMauiApp()
            {
                var builder = MauiApp.CreateBuilder();
                builder
                .UseMauiApp<App>()
                .ConfigureFonts(fonts =>
                {
                    fonts.AddFont("OpenSans-Regular.ttf", "OpenSansRegular");
                    fonts.AddFont("OpenSans-Semibold.ttf", "OpenSansSemibold");
                });
                
                builder.RegisterFirebaseServices();
                
                builder.Services.AddHttpClient("custom-httpclient")
                .ConfigurePrimaryHttpMessageHandler(() =>
                {
                    return new HttpClientHandler
                    {
                        ServerCertificateCustomValidationCallback = (message, cert, chain, errors) => true
                    };
                })
                .ConfigureHttpClient(httpClient =>
                {
                    var baseAddress = DeviceInfo.Platform == DevicePlatform.Android
                    ? "https://10.0.2.2:7194"
                    : "https://localhost:7194";
                    
                    httpClient.BaseAddress = new Uri(baseAddress);
                });
                
                #if DEBUG
                builder.Logging.AddDebug();
                
                builder.Services.AddSingleton<ClientService>();
                builder.Services.AddSingleton<UserService>();
                builder.Services.AddSingleton<LoginPage>();
                builder.Services.AddSingleton<HomePage>();
                builder.Services.AddSingleton<DataPage>();
                
                builder.Services.AddSingleton<LoginPageViewModel>();
                builder.Services.AddSingleton<HomePageViewModel>();
                builder.Services.AddSingleton<DataPageViewModel>();
                #endif
                
                var app = builder.Build();
                
                CrossFirebaseCloudMessaging.Current.NotificationTapped += async (sender, e) =>
                {
                    Console.WriteLine("Notification clicked!");
                    
                    try
                    {
                        await MainThread.InvokeOnMainThreadAsync(async () =>
                        {
                            await Shell.Current.GoToAsync($"//{nameof(DataPage)}");
                        });
                    }
                    catch (Exception ex)
                    {
                        Console.WriteLine($"Navigation failed: {ex.Message}");
                    }
                };
                
                return app;
            }
            
            private static MauiAppBuilder RegisterFirebaseServices(this MauiAppBuilder builder)
            {
                builder.ConfigureLifecycleEvents(events =>
                {
                    #if IOS
                    events.AddiOS(iOS => iOS.WillFinishLaunching((_, __) =>
                    {
                        CrossFirebase.Initialize();
                        FirebaseCloudMessagingImplementation.Initialize();
                        return false;
                    }));
                    #elif ANDROID
                    events.AddAndroid(android => android.OnCreate((activity, _) =>
                    CrossFirebase.Initialize(activity)));
                    #endif
                });
                
                return builder;
            }
        }
    }
\end{lstlisting}
