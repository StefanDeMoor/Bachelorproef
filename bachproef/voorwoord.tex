%%=============================================================================
%% Voorwoord
%%=============================================================================

\chapter*{\IfLanguageName{dutch}{Woord vooraf}{Preface}}%
\label{ch:voorwoord}

%% TODO:
%% Het voorwoord is het enige deel van de bachelorproef waar je vanuit je
%% eigen standpunt (``ik-vorm'') mag schrijven. Je kan hier bv. motiveren
%% waarom jij het onderwerp wil bespreken.
%% Vergeet ook niet te bedanken wie je geholpen/gesteund/... heeft

Voor u ligt mijn bachelorproef, het resultaat van een boeiend onderzoeks- en ontwikkeltraject waarin ik mij heb verdiept in het ontwerpen van een mobiele applicatie voor het SmartKit-systeem van ATS. Deze applicatie heeft als doel de communicatie van systeemmeldingen efficiënter, veiliger en gebruiksvriendelijker te maken via pushnotificaties in plaats van e-mailberichten.

Dit traject was niet alleen technisch uitdagend, maar bood ook de mogelijkheid om diepgaand te werken rond thema’s als mobiele ontwikkeling, authenticatie en systeemintegratie. Dankzij dit project kon ik mijn theoretische kennis in praktijk brengen en ervaring opdoen in het bouwen van een werkende Proof of Concept met behulp van moderne technologieën zoals .NET MAUI en Firebase.

Deze bachelorproef had niet tot stand kunnen komen zonder de hulp en begeleiding van enkele personen, die ik graag wil bedanken. In de eerste plaats wil ik mijn promotor, mevrouw Fien Spriet, oprecht danken voor haar kritische blik, constructieve feedback en voortdurende ondersteuning tijdens het volledige proces. Haar inzichten hielpen me om mijn werk telkens naar een hoger niveau te tillen.

Daarnaast gaat mijn dank uit naar de co-promotor, de heer Glenn Bettens, voor zijn technische expertise, waardevolle suggesties en de mogelijkheid om aan een realistisch en relevant probleem te werken binnen een professionele context. Zijn betrokkenheid en praktijkervaring hebben het onderzoeksproces sterk verrijkt.

Tot slot wil ik ook mijn vriendin en familie bedanken voor hun aanmoediging en steun gedurende het hele academiejaar.