%%=============================================================================
%% Samenvatting
%%=============================================================================

% TODO: De "abstract" of samenvatting is een kernachtige (~ 1 blz. voor een
% thesis) synthese van het document.
%
% Een goede abstract biedt een kernachtig antwoord op volgende vragen:
%
% 1. Waarover gaat de bachelorproef?
% 2. Waarom heb je er over geschreven?
% 3. Hoe heb je het onderzoek uitgevoerd?
% 4. Wat waren de resultaten? Wat blijkt uit je onderzoek?
% 5. Wat betekenen je resultaten? Wat is de relevantie voor het werkveld?
%
% Daarom bestaat een abstract uit volgende componenten:
%
% - inleiding + kaderen thema
% - probleemstelling
% - (centrale) onderzoeksvraag
% - onderzoeksdoelstelling
% - methodologie
% - resultaten (beperk tot de belangrijkste, relevant voor de onderzoeksvraag)
% - conclusies, aanbevelingen, beperkingen
%
% LET OP! Een samenvatting is GEEN voorwoord!

%%---------- Nederlandse samenvatting -----------------------------------------
%
% TODO: Als je je bachelorproef in het Engels schrijft, moet je eerst een
% Nederlandse samenvatting invoegen. Haal daarvoor onderstaande code uit
% commentaar.
% Wie zijn bachelorproef in het Nederlands schrijft, kan dit negeren, de inhoud
% wordt niet in het document ingevoegd.

\IfLanguageName{english}{%
\selectlanguage{dutch}
\chapter*{Samenvatting}
\lipsum[1-4]
\selectlanguage{english}
}{}

%%---------- Samenvatting -----------------------------------------------------
% De samenvatting in de hoofdtaal van het document

\chapter*{\IfLanguageName{dutch}{Samenvatting}{Abstract}}

Dit onderzoek richt zich op de ontwikkeling van een mobiele applicatie voor het energiebeheersysteem SmartKit van ATS, dat energiestromen zoals zonnepanelen, batterijen, laadpalen en verwarming monitort en aanstuurt. Het doel is de huidige communicatie van systeemmeldingen en foutmeldingen, die nu via e-mail plaatsvinden, te verbeteren door deze om te zetten naar mobiele notificaties, zodat gebruikers sneller en efficiënter geïnformeerd worden. Het onderzoek onderzoekt welke technologie het meest geschikt is voor de ontwikkeling van een kostenefficiënte en schaalbare app. Drie benaderingen worden vergeleken: native apps, Progressive Web Apps (PWA), en cross-platform frameworks (zoals React Native en Flutter). \\

De centrale onderzoeksvraag is: "Welke benadering is het meest geschikt voor de ontwikkeling van de mobiele app en hoe kan deze op een veilige manier user management en notificaties integreren?" De methodologie omvat een literatuurstudie over de verschillende technologieën, gevolgd door de ontwikkeling van een Proof of Concept (POC) om de haalbaarheid van de gekozen oplossing te testen. \\

Voor de Proof of Concept werd gekozen om de app te ontwikkelen met het cross-platform framework .NET MAUI, vanwege de goede ondersteuning voor zowel Android als iOS, de integratie met C# (waar het bedrijf al ervaring mee heeft), en de flexibiliteit op vlak van back-end koppeling. In het testproject werden essentiële functies uitgewerkt zoals gebruikersauthenticatie met JWT, veilige opslag via Secure Storage, en pushnotificaties via Firebase. Uit de tests blijkt dat deze technieken niet alleen technisch haalbaar zijn, maar ook vlot werken binnen een mobiele context. Hoewel de POC niet werd geïntegreerd in SmartKit zelf, tonen de resultaten aan dat de gekozen benadering veelbelovend is. Daarmee wordt de centrale onderzoeksvraag beantwoord: .NET MAUI in combinatie met Firebase biedt een veilige en kostenefficiënte oplossing die kan dienen als basis voor verdere integratie in SmartKit. Verdere stappen richting integratie vereisen wel bijkomend werk zoals diepgaandere tests, afstemming met bestaande systemen en gebruikersfeedback.
