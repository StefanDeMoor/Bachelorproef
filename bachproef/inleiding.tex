%%=============================================================================
%% Inleiding
%%=============================================================================

\chapter{\IfLanguageName{dutch}{Inleiding}{Introduction}}%
\label{ch:inleiding}

\noindent In een wereld waarin energiebeheer steeds belangrijker wordt, speelt efficiënte monitoring en besturing van energiestromen een cruciale rol. Het SmartKit-systeem van ATS is ontworpen om deze energiestromen, waaronder zonnepanelen, batterijen, laadpalen en verwarmingssystemen, te beheren. Momenteel worden systeemmeldingen, zoals storingen of operationele statusupdates, via e-mail gecommuniceerd. Dit leidt echter tot een overdaad aan berichten, waardoor kritieke meldingen over het hoofd kunnen worden gezien. Een effectievere oplossing is nodig om real-time meldingen beter te beheren en de respons op systeemstoringen te versnellen. \\

\section{\IfLanguageName{dutch}{Probleemstelling}{Problem Statement}}%
\label{sec:probleemstelling}

\noindent Dit onderzoek richt zich op de ontwikkeling van een mobiele applicatie voor SmartKit die meldingen efficiënter en veiliger naar de juiste gebruikers stuurt via mobiele notificaties. Een dergelijke applicatie kan de beheerders van het systeem in staat stellen sneller en gerichter te reageren, wat de algehele efficiëntie en betrouwbaarheid van het systeem verbetert. \\

\section{\IfLanguageName{dutch}{Onderzoeksvraag}{Research question}}%
\label{sec:onderzoeksvraag}

\noindent De centrale onderzoeksvraag luidt:

\begin{quote}
    \textit{Welke technologie is het meest geschikt voor de ontwikkeling van een mobiele applicatie die systeemmeldingen efficiënt en veilig verstuurt, en welke technologie kan op een kostenefficiënte manier integreren met het bestaande SmartKit-systeem?}
\end{quote}

\noindent Om deze onderzoeksvraag te beantwoorden, worden de volgende deelvragen onderzocht:

\begin{itemize}
    \item Wat zijn de voor- en nadelen van native apps, Progressive Web Apps (PWA) en cross-platform frameworks voor de ontwikkeling van de mobiele applicatie? Hierbij wordt gekeken naar prestaties, gebruikservaring en ontwikkeltijd.
    \item Hoe kan de mobiele applicatie op een veilige manier gebruikers beheren en toegang reguleren? Dit omvat authenticatie en autorisatie om gevoelige systeemdata te beschermen.
    \item Welke technologie biedt de beste mogelijkheden voor het efficiënt versturen van mobiele notificaties binnen het SmartKit-systeem? Hier wordt onderzocht welke pushnotificatietechnologieën geschikt zijn voor real-time communicatie.
    \item Wat zijn de kosten- en schaalvoordelen van de verschillende technologieën voor de ontwikkeling van een mobiele app binnen een klein ontwikkelteam? Dit aspect richt zich op de praktische haalbaarheid en implementatiekosten.
\end{itemize}

\section{\IfLanguageName{dutch}{Onderzoeksdoelstelling}{Research objective}}%
\label{sec:onderzoeksdoelstelling}

\noindent De doelstelling van dit onderzoek is om een technologisch onderbouwde keuze te maken voor de ontwikkeling van een mobiele applicatie die de communicatie van SmartKit-meldingen optimaliseert. Dit zal resulteren in een gedetailleerd onderzoeksrapport en een Proof of Concept (PoC) die de geselecteerde technologieën in de praktijk demonstreert. De uitkomsten van dit onderzoek zullen bijdragen aan een verbeterde werking van het SmartKit-systeem, een efficiëntere afhandeling van systeemmeldingen en een verhoogde klanttevredenheid. \\

\section{\IfLanguageName{dutch}{Opzet van deze bachelorproef}{Structure of this bachelor thesis}}%
\label{sec:opzet-bachelorproef}

\noindent De rest van deze bachelorproef is als volgt opgebouwd:

In Hoofdstuk~\ref{ch:stand-van-zaken} wordt een overzicht gegeven van de stand van zaken binnen het onderzoeksdomein, op basis van een literatuurstudie.

In Hoofdstuk~\ref{ch:methodologie} wordt de methodologie toegelicht en worden de gebruikte onderzoekstechnieken besproken om een antwoord te kunnen formuleren op de onderzoeksvragen.

% TODO: Vul hier aan voor je eigen hoofstukken, één of twee zinnen per hoofdstuk

In Hoofdstuk~\ref{ch:conclusie}, tenslotte, wordt de conclusie gegeven en een antwoord geformuleerd op de onderzoeksvragen. Daarbij wordt ook een aanzet gegeven voor toekomstig onderzoek binnen dit domein.

