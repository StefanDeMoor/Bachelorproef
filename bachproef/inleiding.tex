%%=============================================================================
%% Inleiding
%%=============================================================================

\chapter{\IfLanguageName{dutch}{Inleiding}{Introduction}}%
\label{ch:inleiding}

\noindent
In ons dagelijks leven proberen we bewust om te gaan met energie. We schakelen verlichting uit wanneer we een kamer verlaten en laden elektrische voertuigen op wanneer de omstandigheden gunstig zijn. Al deze kleine keuzes weerspiegelen een bredere maatschappelijke trend waarin energiebeheer een steeds belangrijkere rol speelt. \\

Recente studies en praktijkervaringen bevestigen deze ontwikkeling. Volgens \autocite{Leemburg2025} verandert energiebeheer snel van een wettelijke verplichting naar een strategisch stuurinstrument binnen organisaties. Door de opkomst van variabele energieprijzen, drukte op het elektriciteitsnet en eigen energiebronnen zoals zonneparken of thuisbatterijen, is het belangrijker dan ooit om inzicht te hebben in het energieverbruik en dit actief te kunnen sturen. Zonder gestructureerd energiebeheer blijft men kwetsbaar voor onverwachte kosten, inefficiënties en gemiste duurzaamheidsdoelen. Leemburg benadrukt dat pas wanneer organisaties beschikken over betrouwbare verbruiksdata, ze in staat zijn om daadwerkelijk effectief beleid te voeren rondom energie.\\

\noindent
In deze context is het van cruciaal belang dat energiestromen niet alleen inzichtelijk zijn, maar ook efficiënt worden bewaakt en beheerd. Volgens \textcite{Selleslagh2024} groeit de nood aan gestructureerd energiebeheer sterk, onder meer door de elektrificatie van vervoer en verwarming, de toename van zonnepanelen en thuisbatterijen, en de grotere volatiliteit van energietarieven. 
Het actief spreiden en aansturen van het energieverbruik essentieel om piekbelasting op het elektriciteitsnet te vermijden en om optimaal gebruik te maken van lokaal opgewekte zonne-energie. Met het capaciteitstarief in Vlaanderen, dat het elektriciteitsnetgebruik mee bepaalt op basis van piekverbruik, kan slecht gemonitord of ongecontroleerd energiegedrag leiden tot onnodige kosten. \\

Een energiemanagementsysteem (EMS) speelt hierin een sleutelrol. Zo’n systeem biedt niet alleen realtime inzicht in verbruik en productie, maar kan ook automatisch toestellen aansturen op basis van actuele omstandigheden zoals zoninstraling, netbelasting of dynamische tarieven. Onderzoek van Flux50 toont aan dat huishoudens met een EMS in combinatie met zonnepanelen, batterij en dynamisch contract tot honderden euro’s per jaar kunnen besparen, bovenop mogelijke opbrengsten uit flexibiliteitsdiensten aan het net \autocite{Selleslagh2024}. Daarnaast kunnen EMS'en het comfort verhogen door slimme automatisering, terwijl ze ook bijdragen aan een stabieler elektriciteitsnet.\\

Uit internationale en Belgische praktijkvoorbeelden blijkt dat verschillende bedrijven experimenteren met dergelijke systemen voor het optimaliseren van hun energieverbruik. Zo worden in projecten met onder andere EnergyVille, VITO en imec systemen getest die energiestromen dynamisch aansturen op basis van netbelasting en prijsprikkels. Deze ontwikkelingen tonen aan dat slimme sturing in reële toepassingen tot meetbare efficiëntiewinsten leidt.\\

In deze context ontwikkelde ATS Groep het SmartKit-systeem, een platform dat ontworpen is om diverse energiestromen waaronder zonnepanelen, batterijen, laadpalen en verwarmingssystemen op een geïntegreerde manier te beheren. SmartKit maakt het mogelijk om automatisch verbruiksdata te verzamelen, toestellen aan te sturen en meldingen te genereren op basis van afwijkingen of onderhoudsbehoeften. Momenteel communiceert SmartKit systeemmeldingen, zoals storingen of statusupdates, hoofdzakelijk via e-mail. Dit leidt echter tot een overvloed aan berichten, waardoor kritieke meldingen gemakkelijk over het hoofd worden gezien. Er is daarom behoefte aan een efficiënter systeem voor het realtime beheren en prioriteren van meldingen, zodat afwijkingen sneller kunnen worden opgespoord en verholpen. Een verbeterde meldingsstructuur kan zo niet alleen de bedrijfszekerheid van het systeem verhogen, maar ook de energetische prestaties van het gebouw verbeteren.\\

ATS Groep, oftewel A.T.S. nv, is een multidisciplinaire technologiegroep met hoofdzetel in Merelbeke. De afkorting staat voor \textit{Automatisatie Technieken Schepens}, verwijzend naar oprichter René Schepens. Sinds de oprichting in 1984 groeide het bedrijf uit tot een toonaangevende speler in elektrotechniek, mechatronica, HVAC en energieoplossingen. Met ruim 1200 medewerkers en een geconsolideerde omzet van 365 miljoen euro (2023) biedt ATS totaaloplossingen aan voor industriële, tertiaire en publieke sectoren. In 2015 verwierf energiebedrijf Luminus een meerderheidsparticipatie in ATS Groep, wat de expertise in duurzame en innovatieve energieoplossingen verder versterkte. Met producten zoals SmartKit zet ATS actief in op co-creatie met klanten, energiebesparing en digitalisering van infrastructuurbeheer.


\section{\IfLanguageName{dutch}{Probleemstelling}{Problem Statement}}%
\label{sec:probleemstelling}

\noindent
Dit onderzoek richt zich op de ontwikkeling van een mobiele applicatie voor SmartKit die meldingen efficiënter en veiliger naar de juiste gebruikers stuurt via mobiele notificaties. Een dergelijke applicatie kan de beheerders van het systeem in staat stellen sneller en gerichter te reageren, wat de algehele efficiëntie en betrouwbaarheid van het systeem verbetert.\\

SmartKit is een modulair energiemanagementplatform ontwikkeld door ATS Groep, waarmee bedrijven hun energiestromen zoals zonne- en windenergie, batterijopslag, elektrische laadpalen en HVAC-systemen centraal kunnen monitoren, beheren en automatisch aansturen. Dankzij zijn open architectuur en schaalbare opbouw kan SmartKit flexibel geïntegreerd worden met diverse assets en systemen, zowel in industriële omgevingen als in gebouwbeheer. De software ondersteunt onder meer slimme laadinfrastructuur, batterijoptimalisatie, en dynamische PV-sturing. De kracht van SmartKit zit in het combineren van energiemonitoring en -management, wat bedrijven niet alleen inzicht geeft in hun verbruik, maar hen ook in staat stelt om actief en kostenefficiënt te sturen op basis van realtime gegevens. Deze veelzijdigheid maakt het platform bij uitstek geschikt voor toepassingen waarin snelle detectie en gepaste reacties op afwijkingen cruciaal zijn precies waar dit onderzoek op focust.\\


\section{\IfLanguageName{dutch}{Onderzoeksvraag}{Research question}}%
\label{sec:onderzoeksvraag}

\noindent De centrale onderzoeksvraag luidt:

\begin{quote} 
    \textit{Welke technologie is het meest geschikt voor de ontwikkeling van een mobiele applicatie die systeemmeldingen efficiënt en veilig verstuurt, en hoe kan deze technologie op een kostenefficiënte manier worden geïntegreerd met het bestaande SmartKit-systeem?} 
\end{quote}

\noindent Om deze vraag te beantwoorden, worden verschillende aspecten van mobiele applicatieontwikkeling onderzocht. Dit leidt tot de volgende deelvragen:

\begin{itemize}
    \item Wat zijn de voor- en nadelen van native apps, Progressive Web Apps (PWA) en cross-platform frameworks voor de ontwikkeling van de mobiele applicatie? Hierbij wordt gekeken naar prestaties, gebruikservaring en ontwikkeltijd.
    
    \item Hoe kan de mobiele applicatie op een veilige manier gebruikers beheren en toegang reguleren? Dit omvat aspecten zoals authenticatie en autorisatie, die essentieel zijn om gevoelige systeemdata binnen het SmartKit-systeem te beschermen.
    
    \item Welke technologieën zijn het meest geschikt voor het efficiënt versturen van mobiele notificaties binnen de gekozen ontwikkelstrategie? Hierbij wordt onderzocht welke pushnotificatietechnologieën geschikt zijn voor real-time communicatie en hoe deze kunnen worden geïntegreerd binnen het SmartKit-systeem.
    
    \item Hoe kan een notificatie betrouwbaar worden afgeleverd aan meerdere gebruikers en aan alle toestellen van eenzelfde gebruiker? Hierbij wordt gekeken naar de technische mogelijkheden en beperkingen van notificatiediensten binnen het gekozen ontwikkelplatform, met aandacht voor schaalbaarheid, betrouwbaarheid en efficiënt beheer van notificatiekanalen.
\end{itemize}

\noindent
Op basis van dit onderzoek zal één technologie geselecteerd worden voor verdere uitwerking. De evaluatie van mogelijke frameworks en tools gebeurt in een latere fase.


\section{\IfLanguageName{dutch}{Onderzoeksdoelstelling}{Research objective}}%
\label{sec:onderzoeksdoelstelling}

\noindent De doelstelling van dit onderzoek is om een technologisch onderbouwde keuze te maken voor de ontwikkeling van een mobiele applicatie die de communicatie van SmartKit-meldingen optimaliseert. Dit zal resulteren in een gedetailleerd onderzoeksrapport en een Proof of Concept (PoC) die de geselecteerde technologieën in de praktijk demonstreert. De uitkomsten van dit onderzoek zullen bijdragen aan een verbeterde werking van het SmartKit-systeem, een efficiëntere afhandeling van systeemmeldingen en een verhoogde klanttevredenheid. \\

\section{\IfLanguageName{dutch}{Opzet van deze bachelorproef}{Structure of this bachelor thesis}}%
\label{sec:opzet-bachelorproef}

Deze bachelorproef is opgebouwd uit verschillende hoofdstukken die elk een specifiek deel van het onderzoekstraject behandelen. De structuur is logisch opgebouwd, startend met de context en achtergrondinformatie, en eindigend bij de realisatie en evaluatie van de oplossing.\\

In het eerste hoofdstuk wordt de probleemstelling en onderzoeksvraag toegelicht. Er wordt uitgelegd waarom het huidige systeem van e-mailmeldingen aan vervanging toe is, en wat de doelstellingen zijn van deze bachelorproef.\\

Het tweede hoofdstuk, getiteld \emph{Stand van zaken}, bevat een literatuurstudie waarin dieper wordt ingegaan op de belangrijkste thema’s binnen dit project, zoals authenticatie, beveiliging, en het gebruik van pushnotificaties in mobiele apps. Hierin wordt ook de keuze voor het .NET MAUI-framework gemotiveerd, onderbouwd met recente vergelijkende studies.\\

In het derde hoofdstuk, \emph{Methodologie}, wordt uiteengezet hoe het onderzoek werd aangepakt. De werkwijze wordt opgesplitst in verschillende fasen, zoals het opzetten van de ontwikkelomgeving, het bouwen van een loginpagina, het beveiligen van de sessies via JWT, en het integreren van pushnotificaties via Firebase. Per fase wordt toegelicht welke stappen zijn ondernomen, welke technologieën zijn toegepast en welke resultaten dat heeft opgeleverd.\\

Tot slot worden in het laatste hoofdstuk de belangrijkste conclusies geformuleerd, samen met enkele aanbevelingen voor verder onderzoek en toekomstige implementatie binnen het SmartKit-systeem.\\

Deze opbouw zorgt ervoor dat de lezer op een logische manier wordt meegenomen door het onderzoek, van de theoretische onderbouwing tot de praktische uitvoering en evaluatie van het resultaat.

