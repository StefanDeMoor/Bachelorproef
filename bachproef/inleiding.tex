%%=============================================================================
%% Inleiding
%%=============================================================================

\chapter{\IfLanguageName{dutch}{Inleiding}{Introduction}}%
\label{ch:inleiding}

\noindent In een wereld waarin energiebeheer steeds belangrijker wordt, speelt efficiënte monitoring en besturing van energiestromen een cruciale rol. Het SmartKit-systeem van ATS is ontworpen om deze energiestromen, waaronder zonnepanelen, batterijen, laadpalen en verwarmingssystemen, te beheren. Momenteel worden systeemmeldingen, zoals storingen of operationele statusupdates, via e-mail gecommuniceerd. Dit leidt echter tot een overdaad aan berichten, waardoor kritieke meldingen over het hoofd kunnen worden gezien. Een effectievere oplossing is nodig om real-time meldingen beter te beheren en de respons op systeemstoringen te versnellen. \\

\section{\IfLanguageName{dutch}{Probleemstelling}{Problem Statement}}%
\label{sec:probleemstelling}

\noindent Dit onderzoek richt zich op de ontwikkeling van een mobiele applicatie voor SmartKit die meldingen efficiënter en veiliger naar de juiste gebruikers stuurt via mobiele notificaties. Een dergelijke applicatie kan de beheerders van het systeem in staat stellen sneller en gerichter te reageren, wat de algehele efficiëntie en betrouwbaarheid van het systeem verbetert. \\

\section{\IfLanguageName{dutch}{Onderzoeksvraag}{Research question}}%
\label{sec:onderzoeksvraag}

\noindent De centrale onderzoeksvraag luidt:

\begin{quote} 
    \textit{Welke technologie is het meest geschikt voor de ontwikkeling van een mobiele applicatie die systeemmeldingen efficiënt en veilig verstuurt, en hoe kan deze technologie op een kostenefficiënte manier worden geïntegreerd met het bestaande SmartKit-systeem?} 
\end{quote}

\noindent Om deze vraag te beantwoorden, worden verschillende aspecten van mobiele applicatieontwikkeling onderzocht. Dit leidt tot de volgende deelvragen:

\begin{itemize}
    \item Wat zijn de voor- en nadelen van native apps, Progressive Web Apps (PWA) en cross-platform frameworks voor de ontwikkeling van de mobiele applicatie? Hierbij wordt gekeken naar prestaties, gebruikservaring en ontwikkeltijd.
    
    \item Hoe kan de mobiele applicatie op een veilige manier gebruikers beheren en toegang reguleren? Dit omvat aspecten zoals authenticatie en autorisatie, die essentieel zijn om gevoelige systeemdata binnen het SmartKit-systeem te beschermen.
    
    \item Welke technologieën zijn het meest geschikt voor het efficiënt versturen van mobiele notificaties binnen de gekozen ontwikkelstrategie? Hierbij wordt onderzocht welke pushnotificatietechnologieën geschikt zijn voor real-time communicatie en hoe deze kunnen worden geïntegreerd binnen het SmartKit-systeem.
    
    \item Hoe kan een notificatie betrouwbaar worden afgeleverd aan meerdere gebruikers en aan alle toestellen van eenzelfde gebruiker? Hierbij wordt gekeken naar de technische mogelijkheden en beperkingen van notificatiediensten binnen het gekozen ontwikkelplatform, met aandacht voor schaalbaarheid, betrouwbaarheid en efficiënt beheer van notificatiekanalen.
\end{itemize}

\noindent Op basis van dit onderzoek zal één technologie geselecteerd worden voor verdere uitwerking. Daarbij wordt onder andere .NET MAUI als mogelijke optie verder onderzocht en geëvalueerd.


\section{\IfLanguageName{dutch}{Onderzoeksdoelstelling}{Research objective}}%
\label{sec:onderzoeksdoelstelling}

\noindent De doelstelling van dit onderzoek is om een technologisch onderbouwde keuze te maken voor de ontwikkeling van een mobiele applicatie die de communicatie van SmartKit-meldingen optimaliseert. Dit zal resulteren in een gedetailleerd onderzoeksrapport en een Proof of Concept (PoC) die de geselecteerde technologieën in de praktijk demonstreert. De uitkomsten van dit onderzoek zullen bijdragen aan een verbeterde werking van het SmartKit-systeem, een efficiëntere afhandeling van systeemmeldingen en een verhoogde klanttevredenheid. \\

\section{\IfLanguageName{dutch}{Opzet van deze bachelorproef}{Structure of this bachelor thesis}}%
\label{sec:opzet-bachelorproef}

TODO

