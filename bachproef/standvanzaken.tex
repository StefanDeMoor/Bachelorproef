\chapter{\IfLanguageName{dutch}{Stand van zaken}{State of the art}}%
\label{ch:stand-van-zaken}

% Tip: Begin elk hoofdstuk met een paragraaf inleiding die beschrijft hoe
% dit hoofdstuk past binnen het geheel van de bachelorproef. Geef in het
% bijzonder aan wat de link is met het vorige en volgende hoofdstuk.

% Pas na deze inleidende paragraaf komt de eerste sectiehoofding.
\section{Mobiele Applicatieontwikkeling met .NET MAUI:\\ Focus op Authenticatie en Beveiliging}\\

Mobiele applicatieontwikkeling heeft zich in de afgelopen jaren sterk ontwikkeld, waarbij beveiliging en platformkeuze een grote rol spelen. De noodzaak voor robuuste beveiligingsmaatregelen is groter dan ooit, aangezien mobiele applicaties steeds meer persoonlijke gegevens verwerken. In dit onderzoek wordt de ontwikkeling van een mobiele applicatie met .NET MAUI geanalyseerd, met een focus op authenticatie en beveiliging. Een belangrijk aspect hierbij is de implementatie van JSON Web Tokens (JWT), hashing van wachtwoorden, en het creëren van een veilige loginpagina. Deze methoden worden toegepast om een veilige en efficiënte gebruikerservaring te garanderen, en zijn essentieel voor het beschermen van gevoelige informatie tegen ongeautoriseerde toegang. \\

.NET MAUI (Multi-platform App UI) is gekozen als ontwikkelplatform vanwege de mogelijkheid om een enkele codebase te gebruiken voor zowel Android als iOS. Dit vereenvoudigt de ontwikkeling, onderhoud en distributie van de applicatie. .NET MAUI bouwt voort op Xamarin, maar biedt betere prestaties, een verbeterde gebruikersinterface en diepere integratie met .NET 7/8 auotcite{}. Deze integratie zorgt voor een robuuste backend die goed samenwerkt met andere .NET-technologieën, zoals ASP.NET Core voor server-side processen en Azure voor cloudgebaseerde diensten. Dit maakt het platform bijzonder aantrekkelijk voor ontwikkelaars die reeds ervaring hebben met C# en het .NET-ecosysteem, aangezien het hen in staat stelt om efficiënter te werken en gebruik te maken van hun bestaande kennis autocite{}. \\

Voor de authenticatie is JSON Web Token (JWT) gekozen als oplossing. JWT is een stateless authenticatiemethode die schaalbaarheid verhoogt en de afhankelijkheid van server-side sessiebeheer vermindert autocite{}. In tegenstelling tot traditionele sessiebeheermechanismen, waarbij gegevens op de server worden opgeslagen, bevat een JWT-token zelf de versleutelde gebruikersinformatie. Dit maakt snellere verificatie mogelijk, omdat de server geen sessie-informatie hoeft op te slaan of te raadplegen voor elke gebruikersaanroep. Dit betekent ook dat JWT’s ideaal zijn voor applicaties met veel gelijktijdige gebruikers, zoals sociale netwerken en online marktplaatsen. De beveiliging van JWT-tokens is cruciaal, aangezien een gelekt token kan leiden tot ongeautoriseerde toegang. Om dit te voorkomen, worden ze alleen via HTTPS verzonden en voorzien van een beperkte geldigheidsduur. Dit maakt het moeilijker voor aanvallers om een gestolen token langdurig te misbruiken autocite{}. \\

Om wachtwoorden veilig op te slaan, wordt gebruikgemaakt van bcrypt of PBKDF2, cryptografische hashing-algoritmes die bestand zijn tegen brute-force aanvallen autocite{}. Brute-force aanvallen, waarbij hackers proberen verschillende wachtwoordcombinaties te raden, kunnen effectiever worden afgeslagen door veilige hashing-algoritmes te gebruiken. Daarnaast wordt salting toegepast, een techniek waarbij een willekeurige waarde wordt toegevoegd aan het wachtwoord voordat het gehasht wordt. Dit voorkomt dat veelgebruikte woorden en combinaties uit woordenboeken effectief kunnen worden gebruikt voor aanvallen, wat zogenaamde dictionary-aanvallen voorkomt autocite{}. Dit is een belangrijke maatregel, aangezien studies aantonen dat zwakke hashing-algoritmes, zoals MD5 en SHA1, een veelvoorkomende kwetsbaarheid vormen in mobiele applicaties, en hackers vaak toegang krijgen door gebruik te maken van verouderde encryptiemethoden autocite{}. \\

De loginpagina is een cruciaal onderdeel van de applicatie en vereist een combinatie van gebruiksvriendelijkheid en beveiliging. Het is belangrijk dat gebruikers gemakkelijk toegang krijgen tot hun account zonder dat de beveiliging in het gedrang komt. Gebruikers kunnen zich aanmelden met hun e-mailadres en wachtwoord, waarna validatie plaatsvindt via een backend die JWT-tokens genereert en verzendt. Om aanvallen zoals credential stuffing, waarbij geautomatiseerde systemen massaal proberen in te loggen met gestolen inloggegevens, te voorkomen, wordt een limiet ingesteld op het aantal mislukte inlogpogingen. Dit helpt te voorkomen dat aanvallers onterecht toegang proberen te krijgen door het systeem te overbelasten autocite{}. Daarnaast worden extra beveiligingsmaatregelen overwogen, zoals tweefactorauthenticatie (2FA) via Azure AD of Google, om de veiligheid verder te versterken. Deze technologieën bieden een extra laag van bescherming, doordat ze vereisen dat gebruikers zowel hun wachtwoord als een tijdelijke code, vaak gegenereerd door een app of gestuurd via SMS, invoeren om toegang te krijgen tot hun account autocite{}. 

\begin{figure}
  \centering
  \includegraphics[width=0.8\textwidth]{grail.jpg}
  \caption[Voorbeeld figuur.]{\label{fig:grail}Voorbeeld van invoegen van een figuur. Zorg altijd voor een uitgebreid bijschrift dat de figuur volledig beschrijft zonder in de tekst te moeten gaan zoeken. Vergeet ook je bronvermelding niet!}
\end{figure}

\begin{listing}
  \begin{minted}{python}
    import pandas as pd
    import seaborn as sns

    penguins = sns.load_dataset('penguins')
    sns.relplot(data=penguins, x="flipper_length_mm", y="bill_length_mm", hue="species")
  \end{minted}
  \caption[Voorbeeld codefragment]{Voorbeeld van het invoegen van een codefragment.}
\end{listing}


\begin{table}
  \centering
  \begin{tabular}{lcr}
    \toprule
    \textbf{Kolom 1} & \textbf{Kolom 2} & \textbf{Kolom 3} \\
    $\alpha$         & $\beta$          & $\gamma$         \\
    \midrule
    A                & 10.230           & a                \\
    B                & 45.678           & b                \\
    C                & 99.987           & c                \\
    \bottomrule
  \end{tabular}
  \caption[Voorbeeld tabel]{\label{tab:example}Voorbeeld van een tabel.}
\end{table}

