\chapter{\IfLanguageName{dutch}{Stand van zaken}{State of the art}}%
\label{ch:stand-van-zaken}

% Tip: Begin elk hoofdstuk met een paragraaf inleiding die beschrijft hoe
% dit hoofdstuk past binnen het geheel van de bachelorproef. Geef in het
% bijzonder aan wat de link is met het vorige en volgende hoofdstuk.

% Pas na deze inleidende paragraaf komt de eerste sectiehoofding.
\section{Focus op Authenticatie en Beveiliging}\\

De ontwikkeling van mobiele applicaties heeft de afgelopen jaren aanzienlijke vooruitgang geboekt, waarbij beveiliging en platformkeuze steeds belangrijkere factoren zijn geworden. Mobiele applicaties verwerken in toenemende mate persoonlijke gegevens, wat de noodzaak voor robuuste beveiligingsmaatregelen benadrukt \autocite{build382022}. In recente literatuur wordt gewezen op het belang van een sterke beveiligingsarchitectuur, gezien de toename van datalekken en de complexiteit van moderne dreigingen. Tegelijkertijd blijft de keuze van ontwikkelplatformen een strategische beslissing die invloed heeft op prestaties, onderhoud en compatibiliteit met verschillende besturingssystemen.

Een van de recente ontwikkelingen op het gebied van cross-platform mobiele ontwikkeling is de opkomst van .NET MAUI (Multi-platform App UI). Dit framework bouwt voort op eerdere technologieën zoals Xamarin en integreert nauw met het bredere .NET-ecosysteem, waaronder ASP.NET Core en Azure-diensten \autocite{Klesman2023}. In de literatuur wordt .NET MAUI vaak geprezen vanwege de verbeterde prestaties, moderne gebruikersinterface en de mogelijkheid om met één codebase apps te ontwikkelen voor meerdere platformen. Wanneer vergeleken met andere frameworks zoals React Native en Flutter, worden vooral de diepe integratie met Microsoft-technologieën en de schaalbaarheid als voordelen genoemd \autocite{Kuppan2024}.

Op het gebied van beveiliging signaleren onderzoekers een toenemende voorkeur voor moderne authenticatiemethoden. JSON Web Tokens (JWT) worden vaak genoemd als efficiënte en schaalbare manier om authenticatie en autorisatie te beheren in mobiele applicaties \autocite{Gao2023}. Deze tokens bevatten versleutelde gebruikersinformatie en stellen applicaties in staat om zonder server-side sessiebeheer toch een hoge mate van veiligheid te realiseren. JWT wordt gezien als een robuuste oplossing, mits correct geïmplementeerd, bijvoorbeeld met versleuteling, tijdsbeperking en verzending via beveiligde verbindingen.

Daarnaast blijft de veilige opslag van wachtwoorden een essentieel onderwerp binnen mobiele app-beveiliging. De toepassing van cryptografische hashing-algoritmes zoals bcrypt en PBKDF2 wordt breed aanbevolen vanwege hun weerstand tegen brute-force aanvallen \autocite{Gupta2022}. Het gebruik van salting als aanvullende techniek maakt dictionary-aanvallen moeilijker uitvoerbaar en is volgens de literatuur een noodzakelijke maatregel bij moderne wachtwoordbeveiliging \autocite{Arias2025}. Tegelijkertijd wordt in studies gewezen op de risico’s van verouderde algoritmes zoals MD5 en SHA1, die ondanks hun bekendheid vaak onvoldoende bescherming bieden \autocite{ReesCarter2024}.

De gebruikersinterface, en met name de loginpagina, wordt in de literatuur besproken als een kritiek punt waar gebruiksvriendelijkheid en veiligheid in balans moeten zijn. Beperkingen op het aantal mislukte inlogpogingen en het gebruik van tweefactorauthenticatie (2FA) worden genoemd als effectieve maatregelen om aanvallen zoals credential stuffing tegen te gaan \autocite{Chinnasamy2025, Jurisons2024}. Ook wordt het belang benadrukt van een goede gebruikerservaring, zodat beveiligingsmaatregelen geen hinder vormen voor legitieme gebruikers.\\

Een ander belangrijk onderwerp binnen mobiele applicatieontwikkeling is het gebruik van pushnotificaties. Deze notificaties worden vaak gebruikt om gebruikers snel op de hoogte te brengen van belangrijke gebeurtenissen, bijvoorbeeld wanneer een app gekoppeld is aan systemen zoals zonnepanelen of slimme meters. In zulke situaties kunnen pushnotificaties gebruikers waarschuwen voor overproductie van energie of andere relevante veranderingen.

Bij het implementeren van pushnotificaties is een belangrijk technisch aspect het correct registreren en beheren van devicetokens. Voor iOS-apparaten kan worden gebruikgemaakt van Apple Push Notification Service (APNs), terwijl Android-apparaten kunnen gebruikmaken van Firebase Cloud Messaging (FCM). Bij het opstarten van een mobiele applicatie dient elk apparaat een uniek push-token aan te vragen. In het geval van Firebase gebeurt dit door per gebruiker een token te registreren dat aan zijn account wordt gekoppeld. Dit stelt de backend in staat om gerichte notificaties te versturen op basis van gebruikersacties of systeemgebeurtenissen.

Een aandachtspunt hierbij is de levensduur van deze tokens. Pushnotificatietokens kunnen vervallen, bijvoorbeeld bij herinstallatie van de app, verandering van instellingen of het verlopen van de sessie. Daarom is het van belang dat de applicatie in staat is om deze tokens automatisch te verversen en opnieuw te registreren bij de backend. In dit kader wordt vaak ook het automatisch vernieuwen van JWT-tokens op de achtergrond toegepast, zodat gebruikerssessies geldig blijven zonder handmatige interventie. Dit verhoogt niet alleen de veiligheid, maar ook de gebruikerservaring. Verder is het aan te raden om bij registratie het apparaattype (iOS of Android) mee te geven, zodat het systeem het juiste distributiekanaal voor notificaties gebruikt.

In de praktijk zullen gebruikers echter niet continu actief zijn binnen de mobiele applicatie. Dit betekent dat het geldig blijven van het pushnotificatietoken essentieel is: wanneer een token is vervallen of ongeldig is geworden, ontvangt de gebruiker geen meldingen meer – zelfs niet bij kritieke gebeurtenissen zoals het overschot aan opgewekte zonne-energie. In applicaties die afhankelijk zijn van realtime communicatie over belangrijke systeemstatussen, kan dit leiden tot functionele beperkingen of gemiste waarschuwingen.

Daarnaast moet bijzondere aandacht worden besteed aan de beveiliging van pushnotificatietokens en de inhoud van de berichten zelf. Pushnotificaties moeten voldoende gegevens bevatten om na ontvangst direct te kunnen navigeren naar de juiste pagina binnen de app, specifiek afgestemd op de juiste gebruiker en situatie. Dit betekent dat de notificatie bijvoorbeeld een gebruikers-ID of andere contextuele informatie moet bevatten. Omdat het hier vaak gaat om gevoelige gegevens, zoals energieverbruik of systeemstoringen, is het essentieel dat deze informatie op een veilige manier wordt verwerkt en verzonden, bijvoorbeeld via versleuteling of in combinatie met server-side verificatie bij het openen van de app via een notificatie. Een fout in dit proces kan leiden tot datalekken of ongeautoriseerde toegang tot persoonlijke informatie.

Wanneer een token vervalt, kan dit doorgaans worden opgevangen door de pushservice zelf, die een foutcode retourneert wanneer een bericht niet kan worden afgeleverd. Op basis hiervan kan het systeem het oude token verwijderen en bij de eerstvolgende gebruikersinteractie een nieuw token opvragen. Het implementeren van deze logica is essentieel om te waarborgen dat pushnotificaties betrouwbaar blijven functioneren over de gehele levenscyclus van de applicatie.



