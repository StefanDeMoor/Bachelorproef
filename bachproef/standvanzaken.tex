\chapter{\IfLanguageName{dutch}{Stand van zaken}{State of the art}}%
\label{ch:stand-van-zaken}

% Tip: Begin elk hoofdstuk met een paragraaf inleiding die beschrijft hoe
% dit hoofdstuk past binnen het geheel van de bachelorproef. Geef in het
% bijzonder aan wat de link is met het vorige en volgende hoofdstuk.

% Pas na deze inleidende paragraaf komt de eerste sectiehoofding.
\section{Focus op Authenticatie en Beveiliging}\\

De ontwikkeling van mobiele applicaties heeft de afgelopen jaren aanzienlijke vooruitgang geboekt, waarbij beveiliging en platformkeuze steeds belangrijkere factoren zijn geworden. Mobiele applicaties verwerken in toenemende mate persoonlijke gegevens, wat de noodzaak voor robuuste beveiligingsmaatregelen benadrukt \autocite{build382022}. In recente literatuur wordt gewezen op het belang van een sterke beveiligingsarchitectuur, gezien de toename van datalekken en de complexiteit van moderne dreigingen. Tegelijkertijd blijft de keuze van ontwikkelplatformen een strategische beslissing die invloed heeft op prestaties, onderhoud en compatibiliteit met verschillende besturingssystemen.

Een van de recente ontwikkelingen op het gebied van cross-platform mobiele ontwikkeling is de opkomst van .NET MAUI (Multi-platform App UI). Dit framework bouwt voort op eerdere technologieën zoals Xamarin en integreert nauw met het bredere .NET-ecosysteem, waaronder ASP.NET Core en Azure-diensten \autocite{Klesman2023}. In de literatuur wordt .NET MAUI vaak geprezen vanwege de verbeterde prestaties, moderne gebruikersinterface en de mogelijkheid om met één codebase apps te ontwikkelen voor meerdere platformen. Wanneer vergeleken met andere frameworks zoals React Native en Flutter, worden vooral de diepe integratie met Microsoft-technologieën en de schaalbaarheid als voordelen genoemd \autocite{Kuppan2024}.

Op het gebied van beveiliging signaleren onderzoekers een toenemende voorkeur voor moderne authenticatiemethoden. JSON Web Tokens (JWT) worden vaak genoemd als efficiënte en schaalbare manier om authenticatie en autorisatie te beheren in mobiele applicaties \autocite{Gao2023}. Deze tokens bevatten versleutelde gebruikersinformatie en stellen applicaties in staat om zonder server-side sessiebeheer toch een hoge mate van veiligheid te realiseren. JWT wordt gezien als een robuuste oplossing, mits correct geïmplementeerd, bijvoorbeeld met versleuteling, tijdsbeperking en verzending via beveiligde verbindingen.

Daarnaast blijft de veilige opslag van wachtwoorden een essentieel onderwerp binnen mobiele app-beveiliging. De toepassing van cryptografische hashing-algoritmes zoals bcrypt en PBKDF2 wordt breed aanbevolen vanwege hun weerstand tegen brute-force aanvallen \autocite{Gupta2022}. Het gebruik van salting als aanvullende techniek maakt dictionary-aanvallen moeilijker uitvoerbaar en is volgens de literatuur een noodzakelijke maatregel bij moderne wachtwoordbeveiliging \autocite{Arias2025}. Tegelijkertijd wordt in studies gewezen op de risico’s van verouderde algoritmes zoals MD5 en SHA1, die ondanks hun bekendheid vaak onvoldoende bescherming bieden \autocite{ReesCarter2024}.

De gebruikersinterface, en met name de loginpagina, wordt in de literatuur besproken als een kritiek punt waar gebruiksvriendelijkheid en veiligheid in balans moeten zijn. Beperkingen op het aantal mislukte inlogpogingen en het gebruik van tweefactorauthenticatie (2FA) worden genoemd als effectieve maatregelen om aanvallen zoals credential stuffing tegen te gaan \autocite{Chinnasamy2025, Jurisons2024}. Ook wordt het belang benadrukt van een goede gebruikerservaring, zodat beveiligingsmaatregelen geen hinder vormen voor legitieme gebruikers.\\

Een ander belangrijk onderwerp binnen mobiele applicatieontwikkeling is het gebruik van pushnotificaties. Deze notificaties worden vaak gebruikt om gebruikers snel op de hoogte te brengen van belangrijke gebeurtenissen, bijvoorbeeld wanneer een app gekoppeld is aan systemen zoals zonnepanelen of slimme meters. In zulke situaties kunnen pushberichten gebruikers waarschuwen voor overproductie van energie of andere relevante veranderingen.

Een technologie die vaak genoemd wordt voor het versturen van pushnotificaties is Firebase Cloud Messaging (FCM) van Google. Dit systeem maakt het mogelijk om berichten te sturen naar mobiele apparaten op basis van bepaalde acties, tijdstippen of gegevens. Firebase ondersteunt zowel Android als iOS en kan gekoppeld worden aan backenddiensten om meldingen automatisch te versturen. In de literatuur wordt benadrukt dat het belangrijk is om notificaties veilig en betrouwbaar te verzenden. Dit vraagt onder andere om het goed registreren van apparaten en het beheren van wie welke meldingen ontvangt.

\begin{listing}
  \begin{minted}{python}
    import pandas as pd
    import seaborn as sns

    penguins = sns.load_dataset('penguins')
    sns.relplot(data=penguins, x="flipper_length_mm", y="bill_length_mm", hue="species")
  \end{minted}
  \caption[Voorbeeld codefragment]{Voorbeeld van het invoegen van een codefragment.}
\end{listing}


\begin{table}
  \centering
  \begin{tabular}{lcr}
    \toprule
    \textbf{Kolom 1} & \textbf{Kolom 2} & \textbf{Kolom 3} \\
    $\alpha$         & $\beta$          & $\gamma$         \\
    \midrule
    A                & 10.230           & a                \\
    B                & 45.678           & b                \\
    C                & 99.987           & c                \\
    \bottomrule
  \end{tabular}
  \caption[Voorbeeld tabel]{\label{tab:example}Voorbeeld van een tabel.}
\end{table}

