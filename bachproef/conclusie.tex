%%=============================================================================
%% Conclusie
%%=============================================================================

\chapter{Conclusie}%
\label{ch:conclusie}

% TODO: Trek een duidelijke conclusie, in de vorm van een antwoord op de
% onderzoeksvra(a)g(en). Wat was jouw bijdrage aan het onderzoeksdomein en
% hoe biedt dit meerwaarde aan het vakgebied/doelgroep? 
% Reflecteer kritisch over het resultaat. In Engelse teksten wordt deze sectie
% ``Discussion'' genoemd. Had je deze uitkomst verwacht? Zijn er zaken die nog
% niet duidelijk zijn?
% Heeft het onderzoek geleid tot nieuwe vragen die uitnodigen tot verder 
%onderzoek?

Het onderzoek toont aan dat het technisch mogelijk is om een veilige en efficiënte mobiele applicatie te ontwikkelen voor het SmartKit-systeem, gebruikmakend van het .NET MAUI cross-platform framework. De methodologie werd succesvol gevolgd en leidde tot een functionerende Proof of Concept waarin authenticatie via JWT, veilige opslag, en pushnotificaties met Firebase werkten zoals verwacht.

Hoewel de integratie met het bestaande SmartKit-systeem nog niet volledig is uitgevoerd, bewijst de POC dat de gekozen technologieën niet alleen voldoen aan de gestelde functionele en beveiligingseisen, maar ook significant gebruiksvriendelijker en sneller zijn in vergelijking met de bestaande e-mailnotificaties. Dit levert dus een duidelijke meerwaarde op voor zowel eindgebruikers als systeembeheerders.

De bijdrage van dit onderzoek aan het vakgebied ligt in het aantonen dat .NET MAUI een haalbare en kostenefficiënte oplossing biedt voor kleinschalige teams die een moderne mobiele gebruikerservaring willen integreren in bestaande energiesystemen. Daarnaast biedt het project een concreet voorbeeld van hoe real-time communicatie en veilige authenticatie gecombineerd kunnen worden in één mobiel platform.

Toch zijn er nog uitdagingen. De daadwerkelijke integratie met de SmartKit backend vereist bijkomend werk, waaronder uitgebreide testen en afstemming met bestaande infrastructuur. Verder roept het onderzoek nieuwe vragen op, zoals: hoe schaalbaar is de oplossing bij honderden of duizenden gebruikers? En hoe kunnen notificaties dynamisch worden afgestemd op gebruikersvoorkeuren of prioriteit?

