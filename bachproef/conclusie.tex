%%=============================================================================
%% Conclusie
%%=============================================================================

\chapter{Conclusie}%
\label{ch:conclusie}

% TODO: Trek een duidelijke conclusie, in de vorm van een antwoord op de
% onderzoeksvra(a)g(en). Wat was jouw bijdrage aan het onderzoeksdomein en
% hoe biedt dit meerwaarde aan het vakgebied/doelgroep? 
% Reflecteer kritisch over het resultaat. In Engelse teksten wordt deze sectie
% ``Discussion'' genoemd. Had je deze uitkomst verwacht? Zijn er zaken die nog
% niet duidelijk zijn?
% Heeft het onderzoek geleid tot nieuwe vragen die uitnodigen tot verder 
%onderzoek?

Tijdens dit onderzoek werd nagegaan of het mogelijk is om een veilige en efficiënte mobiele applicatie te ontwikkelen voor het SmartKit-systeem. Hiervoor werd gekozen om te werken met .NET MAUI, een cross-platform framework waarmee één codebase gebruikt kan worden voor zowel Android als iOS. In de loop van het project is een Proof of Concept (PoC) opgebouwd, waarin enkele belangrijke functies succesvol zijn geïmplementeerd: gebruikers kunnen zich aanmelden via JWT-authenticatie, hun gegevens worden veilig opgeslagen, en er worden pushnotificaties verstuurd met behulp van Firebase Cloud Messaging (FCM).\\

Naast het technische luik werd ook onderzocht hoe meldingen naar meerdere gebruikers en toestellen kunnen worden gestuurd. De testopstelling toont aan dat FCM hier goede ondersteuning voor biedt. Elk toestel krijgt een uniek token, en er kunnen groepen of onderwerpen worden ingesteld om gerichte meldingen te versturen. Hierdoor is het mogelijk om één gebruiker op meerdere toestellen tegelijk te bereiken, of verschillende gebruikers specifieke meldingen te sturen. In de praktijk vraagt dit natuurlijk om een goed systeem voor tokenbeheer, zeker wanneer er veel gebruikers of toestellen zijn.\\

De PoC is nog niet volledig geïntegreerd met de backend van het SmartKit-systeem, maar toont wel duidelijk aan dat de gekozen aanpak werkt en dat deze een flinke verbetering betekent tegenover het huidige e-mailsysteem. Meldingen komen sneller aan, zijn beter afgestemd op de gebruiker en vallen meer op, wat de kans vergroot dat er snel wordt ingegrepen bij bijvoorbeeld een storing.\\

Uit dit onderzoek blijkt dat .NET MAUI een haalbare en kostenefficiënte keuze is voor kleine teams die op zoek zijn naar een moderne oplossing voor mobiele apps met notificaties. Bovendien biedt dit project een voorbeeld van hoe je real-time communicatie en veiligheid op een praktische manier kunt combineren in een mobiele toepassing voor energiebeheer of gelijkaardige sectoren. Daarmee is de centrale onderzoeksvraag beantwoord: .NET MAUI in combinatie met FCM vormt een veilige, efficiënte en kosteneffectieve keuze die kan worden geïntegreerd in het SmartKit-systeem.\\

Toch blijven er nog enkele uitdagingen. De koppeling met de bestaande infrastructuur moet verder uitgewerkt worden, en ook de schaalbaarheid bij grotere gebruikersgroepen is nog niet volledig getest. In de toekomst zou het interessant zijn om ook te bekijken hoe meldingen kunnen worden gepersonaliseerd op basis van gebruikersinstellingen of prioriteit van het bericht. Dit kan de relevantie van de notificaties nog verder verhogen.\\

Tegelijkertijd moet de uitkomst van dit onderzoek met enige voorzichtigheid worden geïnterpreteerd. De oplossing is tot nu toe enkel getest in een beperkte, gecontroleerde omgeving, waardoor nog niet bewezen is hoe het systeem zich gedraagt in een grootschalige productieomgeving. Daarnaast brengt de keuze voor FCM een zekere afhankelijkheid van een externe leverancier met zich mee, wat op lange termijn beperkingen kan opleveren rond flexibiliteit en databeheer. Ook de gebruikerservaring en het praktische beheer van meldingen zijn nog niet uitgebreid onderzocht. Hoe systeembeheerders en eindgebruikers de notificaties daadwerkelijk zullen ervaren, blijft voorlopig een open vraag. Deze aandachtspunten bieden belangrijke aanknopingspunten voor vervolgonderzoek en verdere verfijning van de oplossing.\\

Met deze resultaten levert het onderzoek niet alleen een praktisch toepasbare oplossing voor ATS Groep, maar draagt het ook bij aan het bredere onderzoeksdomein van energiemanagementsystemen. Het toont hoe mobiele technologie en real-time communicatie kunnen worden ingezet om de efficiëntie, betrouwbaarheid en gebruikerservaring in dit snel evoluerende vakgebied te verbeteren.



