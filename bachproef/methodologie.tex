%%=============================================================================
%% Methodologie
%%=============================================================================

\chapter{\IfLanguageName{dutch}{Methodologie}{Methodology}}%
\label{ch:methodologie}

%% TODO: In dit hoofstuk geef je een korte toelichting over hoe je te werk bent
%% gegaan. Verdeel je onderzoek in grote fasen, en licht in elke fase toe wat
%% de doelstelling was, welke deliverables daar uit gekomen zijn, en welke
%% onderzoeksmethoden je daarbij toegepast hebt. Verantwoord waarom je
%% op deze manier te werk gegaan bent.
%% 
%% Voorbeelden van zulke fasen zijn: literatuurstudie, opstellen van een
%% requirements-analyse, opstellen long-list (bij vergelijkende studie),
%% selectie van geschikte tools (bij vergelijkende studie, "short-list"),
%% opzetten testopstelling/PoC, uitvoeren testen en verzamelen
%% van resultaten, analyse van resultaten, ...
%%
%% !!!!! LET OP !!!!!
%%
%% Het is uitdrukkelijk NIET de bedoeling dat je het grootste deel van de corpus
%% van je bachelorproef in dit hoofstuk verwerkt! Dit hoofdstuk is eerder een
%% kort overzicht van je plan van aanpak.
%%
%% Maak voor elke fase (behalve het literatuuronderzoek) een NIEUW HOOFDSTUK aan
%% en geef het een gepaste titel.

Dit onderzoek werd uitgevoerd in drie fasen, waarbij elke fase een specifieke doelstelling had en resulteerde in concrete deliverables. De gekozen aanpak is gebaseerd op een iteratieve en stapsgewijze ontwikkeling van een mobiele applicatie met \texttt{.NET MAUI}, waarbij zowel frontend als backendcomponenten zijn ontwikkeld en getest. In dit hoofdstuk worden de verschillende fasen toegelicht en wordt verantwoord waarom deze aanpak is gekozen.

\section{Omgeving klaarzetten}

De eerste stap in het onderzoek was het opzetten van de ontwikkelomgeving. Dit was noodzakelijk om een stabiele en efficiënte werkomgeving te creëren waarin de mobiele applicatie kon worden ontwikkeld en getest. Hiervoor werd \texttt{.NET MAUI} als ontwikkelplatform geïnstalleerd en geconfigureerd. Daarnaast werden verschillende simulators opgezet om de applicatie op meerdere apparaten te testen. Tot slot werd een database ingericht om gebruikersgegevens en authenticatie-informatie op te slaan. De gekozen methode voor deze fase bestond uit het raadplegen van de officiële documentatie van \texttt{.NET MAUI} en het uitvoeren van praktische experimenten om de configuratie optimaal af te stemmen. Het resultaat van deze fase was een volledig geconfigureerde ontwikkelomgeving, inclusief de benodigde tools, frameworks en databases.

\section{Loginpagina}

In de tweede fase werd de basisfunctionaliteit voor gebruikersauthenticatie geïmplementeerd. De focus lag op het ontwerpen en ontwikkelen van een gebruiksvriendelijke loginpagina. Hiervoor werd eerst de frontend ontwikkeld in \texttt{.NET MAUI}, waarbij gebruik werd gemaakt van standaard UI-componenten en best practices op het gebied van gebruiksvriendelijkheid. Vervolgens werd een backend opgezet waarin gebruikersgegevens werden beheerd en gecontroleerd via een database. Na de implementatie werd de loginpagina getest op verschillende apparaten om de correcte werking te garanderen. De onderzoeksmethode in deze fase bestond uit een combinatie van literatuurstudie en experimentele ontwikkeling, waarbij verschillende implementaties werden vergeleken en geëvalueerd. Het eindresultaat was een werkende loginpagina die correct communiceert met de backend en gebruikersgegevens valideert.

\section{JWT Authenticatie en Hashing}

De derde fase richtte zich op het beveiligen van de gebruikersauthenticatie door middel van JWT (JSON Web Token) en hashing-technieken. Dit was een cruciale stap om ongeautoriseerde toegang te voorkomen en gebruikersgegevens veilig op te slaan. JWT werd geïmplementeerd als mechanisme voor sessiebeheer, waarbij gebruikers een token ontvangen na succesvolle authenticatie. Dit token wordt vervolgens gebruikt om toegang te krijgen tot beveiligde endpoints. Daarnaast werden wachtwoorden gehasht opgeslagen in de database, zodat ze niet in platte tekst beschikbaar zijn. De beveiligingsmechanismen werden uitvoerig getest om mogelijke zwakke plekken te identificeren en te verbeteren. Ook in deze fase werd een combinatie van literatuurstudie en experimentele ontwikkeling toegepast, waarbij best practices op het gebied van authenticatie en beveiliging werden onderzocht en geïmplementeerd. Het resultaat was een veilige en betrouwbare loginomgeving die voldoet aan moderne beveiligingsstandaarden.

\section{Vervolg...}







