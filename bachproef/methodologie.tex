%%=============================================================================
%% Methodologie
%%=============================================================================

\chapter{\IfLanguageName{dutch}{Methodologie}{Methodology}}%
\label{ch:methodologie}

%% TODO: In dit hoofstuk geef je een korte toelichting over hoe je te werk bent
%% gegaan. Verdeel je onderzoek in grote fasen, en licht in elke fase toe wat
%% de doelstelling was, welke deliverables daar uit gekomen zijn, en welke
%% onderzoeksmethoden je daarbij toegepast hebt. Verantwoord waarom je
%% op deze manier te werk gegaan bent.
%% 
%% Voorbeelden van zulke fasen zijn: literatuurstudie, opstellen van een
%% requirements-analyse, opstellen long-list (bij vergelijkende studie),
%% selectie van geschikte tools (bij vergelijkende studie, "short-list"),
%% opzetten testopstelling/PoC, uitvoeren testen en verzamelen
%% van resultaten, analyse van resultaten, ...
%%
%% !!!!! LET OP !!!!!
%%
%% Het is uitdrukkelijk NIET de bedoeling dat je het grootste deel van de corpus
%% van je bachelorproef in dit hoofstuk verwerkt! Dit hoofdstuk is eerder een
%% kort overzicht van je plan van aanpak.
%%
%% Maak voor elke fase (behalve het literatuuronderzoek) een NIEUW HOOFDSTUK aan
%% en geef het een gepaste titel.

Dit onderzoek werd uitgevoerd in drie fasen, waarbij elke fase een specifieke doelstelling had en resulteerde in concrete deliverables. De gekozen aanpak is gebaseerd op een iteratieve en stapsgewijze ontwikkeling van een mobiele applicatie met \texttt{.NET MAUI}, waarbij zowel frontend als backendcomponenten zijn ontwikkeld en getest. In dit hoofdstuk worden de verschillende fasen toegelicht en wordt verantwoord waarom deze aanpak is gekozen.

\section{Omgeving klaarzetten}

De eerste stap in het onderzoek was het opzetten van de ontwikkelomgeving. Dit was noodzakelijk om een stabiele en efficiënte werkomgeving te creëren waarin de mobiele applicatie kon worden ontwikkeld en getest. Hiervoor werd \texttt{.NET MAUI} als ontwikkelplatform geïnstalleerd en geconfigureerd. Daarnaast werden verschillende simulators opgezet om de applicatie op meerdere apparaten te testen. Tot slot werd een database ingericht om gebruikersgegevens en authenticatie-informatie op te slaan. De gekozen methode voor deze fase bestond uit het raadplegen van de officiële documentatie van \texttt{.NET MAUI} en het uitvoeren van praktische experimenten om de configuratie optimaal af te stemmen. Het resultaat van deze fase was een volledig geconfigureerde ontwikkelomgeving, inclusief de benodigde tools, frameworks en databases.

\section{Loginpagina}

In de tweede fase werd de basisfunctionaliteit voor gebruikersauthenticatie geïmplementeerd. De focus lag op het ontwerpen en ontwikkelen van een gebruiksvriendelijke loginpagina. Hiervoor werd eerst de frontend ontwikkeld in \texttt{.NET MAUI}, waarbij gebruik werd gemaakt van standaard UI-componenten en best practices op het gebied van gebruiksvriendelijkheid. Vervolgens werd een backend opgezet waarin gebruikersgegevens werden beheerd en gecontroleerd via een database. Na de implementatie werd de loginpagina getest op verschillende apparaten om de correcte werking te garanderen. Tijdens deze fase werd gebruikgemaakt van officiële documentatie en technische bronnen om de implementatie te ondersteunen. Verschillende oplossingen zijn onderzocht, toegepast en geëvalueerd om tot een stabiele en veilige authenticatieoplossing te komen. Het eindresultaat was een werkende loginpagina die correct communiceert met de backend en gebruikersgegevens valideert.

\section{JWT Authenticatie en Hashing}

De derde fase richtte zich op het beveiligen van de gebruikersauthenticatie door middel van JWT (JSON Web Token) en hashing-technieken. Dit was een cruciale stap om ongeautoriseerde toegang te voorkomen en gebruikersgegevens veilig op te slaan. JWT werd geïmplementeerd als mechanisme voor sessiebeheer, waarbij gebruikers een token ontvangen na succesvolle authenticatie. Dit token wordt vervolgens gebruikt om toegang te krijgen tot beveiligde endpoints. Daarnaast werden wachtwoorden gehasht opgeslagen in de database, zodat ze niet in platte tekst beschikbaar zijn. De beveiligingsmechanismen werden uitvoerig getest om mogelijke zwakke plekken te identificeren en te verbeteren. Het resultaat was een veilige en betrouwbare loginomgeving die voldoet aan moderne beveiligingsstandaarden.
Om het sessiebeheer verder te versterken, werd ervoor gekozen om het JWT-token niet op te slaan in een lokale database, maar in de secure storage van het mobiele besturingssysteem. Deze opslaglocatie is speciaal ontworpen voor het veilig bewaren van gevoelige informatie en biedt verhoogde bescherming tegen ongeautoriseerde toegang of reverse engineering van de applicatie. Een bijkomend voordeel van deze aanpak is dat het token automatisch beschikbaar blijft zolang de gebruiker is aangemeld, zonder dat een externe databank geconsulteerd hoeft te worden. Bij uitloggen wordt het token expliciet verwijderd, waardoor de toegang volledig wordt afgesloten. Deze methode sluit goed aan bij de vereisten van mobiele applicatiebeveiliging en verhoogt zowel de gebruiksvriendelijkheid als de veiligheid van de app.


\section{Pushnotificaties}

In deze fase werd de mogelijkheid toegevoegd om pushnotificaties te ontvangen en te verzenden binnen de mobiele applicatie. Het doel hiervan is om gebruikers automatisch op de hoogte te brengen van belangrijke updates, bijvoorbeeld wanneer hun systeem een overschot aan zonne-energie detecteert. Om dit te realiseren, werd de applicatie zodanig ingesteld dat elk toestel een unieke code (token) ontvangt zodra de app wordt geopend. Deze token wordt vervolgens gekoppeld aan de juiste gebruiker, zodat meldingen persoonlijk en correct afgeleverd kunnen worden.
Daarnaast werd er gezorgd dat de app, wanneer er op een melding wordt geklikt, automatisch naar de juiste pagina navigeert. Dit is belangrijk omdat gebruikers zo direct toegang krijgen tot de relevante informatie zonder handmatig door de applicatie te moeten zoeken.
Voor het testen van de pushnotificaties werd gebruikgemaakt van het platform Postman. Hiermee konden handmatig HTTP-verzoeken naar Firebase worden gestuurd, waarmee notificaties naar specifieke toestellen werden verzonden. Dit liet toe om de correcte werking van de Firebase-configuratie te controleren, nog vóór de volledige applicatie was geïntegreerd. Bovendien bood deze aanpak de mogelijkheid om meldingen te testen op meerdere apparaten – iets wat met een enkele emulator moeilijk te realiseren is. Door Postman als testtool in te zetten, kon het notificatiesysteem grondig en efficiënt gevalideerd worden.
Tot slot werd de algemene werking getest op verschillende toestellen en werd gecontroleerd of de notificaties ook aankwamen wanneer de app niet actief was. Deze aanpak zorgde ervoor dat de meldingen betrouwbaar en gebruiksvriendelijk functioneren binnen de volledige gebruikerservaring.


\section{Resultaten en Integratie}

Nu de verschillende technologieën en technieken succesvol getest zijn in een kleiner, losstaand project, kunnen we besluiten dat ze in principe goed werken. Denk hierbij aan zaken zoals JWT-authenticatie, veilige opslag van tokens en pushnotificaties via Firebase. De testresultaten waren positief en tonen aan dat deze methodes technisch haalbaar zijn en ook goed functioneren binnen een mobiele applicatie.

Hoewel alles in deze bachelorproef opgebouwd en getest werd in een aparte proof of concept, is het de bedoeling dat deze aanpak op termijn geïntegreerd wordt in SmartKit, de bestaande tool van het bedrijf. Binnen de scope van dit onderzoek lag de focus echter op het uitvoeren van een verkennend onderzoek en het ontwikkelen van een eerste concept. Een volledige integratie in SmartKit zelf zat dus niet in de opdracht, aangezien dat meer tijd, afstemming en grondige testing vraagt.

Toch tonen de behaalde resultaten aan dat de implementatie van deze technologieën in Smartkit zeker mogelijk is en ook een duidelijke meerwaarde zou kunnen bieden. Het vormt dan ook een goede basis voor verdere ontwikkeling en integratie in de toekomst.






