%==============================================================================
% Sjabloon onderzoeksvoorstel bachproef
%==============================================================================
% Gebaseerd op document class `hogent-article'
% zie <https://github.com/HoGentTIN/latex-hogent-article>

% Voor een voorstel in het Engels: voeg de documentclass-optie [english] toe.
% Let op: kan enkel na toestemming van de bachelorproefcoördinator!
\documentclass{hogent-article}
\usepackage{fontspec}
\usepackage{enumitem}
\setmainfont{Times New Roman}  % Ensure this font is installed

% Invoegen bibliografiebestand
\addbibresource{literatuur.bib}

% Informatie over de opleiding, het vak en soort opdracht
\studyprogramme{Professionele bachelor toegepaste informatica}
\course{Bachelorproef}
\assignmenttype{Onderzoeksvoorstel}
% Voor een voorstel in het Engels, haal de volgende 3 regels uit commentaar
% \studyprogramme{Bachelor of applied information technology}
% \course{Bachelor thesis}
% \assignmenttype{Research proposal}

\academicyear{2024-2025} % TODO: pas het academiejaar aan

% TODO: Werktitel
\title{Onderzoek en concept ontwikkeling mobile app voor een Energy Management System.}

% TODO: Studentnaam en emailadres invullen
\author{Stefan De Moor}
\email{stefan.demoor@student.hogent.be}

% TODO: Medestudent
% Gaat het om een bachelorproef in samenwerking met een student in een andere
% opleiding? Geef dan de naam en emailadres hier
% \author{Yasmine Alaoui (naam opleiding)}
% \email{yasmine.alaoui@student.hogent.be}

% TODO: Geef de co-promotor op
\supervisor[Co-promotor]{Glenn Bettens (ATS Groep, \href{mailto:glenn.bettens@atsgroep.be}{glenn.bettens@atsgroep.be})}

% Binnen welke specialisatierichting uit 3TI situeert dit onderzoek zich?
% Kies uit deze lijst:
%
% - Mobile \& Enterprise development
% - AI \& Data Engineering
% - Functional \& Business Analysis
% - System \& Network Administrator
% - Mainframe Expert
% - Als het onderzoek niet past binnen een van deze domeinen specifieer je deze
%   zelf
%
\specialisation{Mobile \& Enterprise development}
\keywords{Scheme, World Wide Web, $\lambda$-calculus}

\begin{document}

\begin{abstract}
  Dit onderzoek richt zich op de ontwikkeling van een mobiele applicatie voor het energiebeheersysteem SmartKit van ATS, dat energiestromen zoals zonnepanelen, batterijen, laadpalen en verwarming zowel monitort als aanstuurt. De uitdaging is om de communicatie van systeemmeldingen en foutmeldingen, die momenteel via e-mail plaatsvinden, te verbeteren door deze om te zetten naar mobiele notificaties die gebruikers sneller en efficiënter informeren. Het specifieke probleem dat in dit onderzoek wordt behandeld, is welke technologie het meest geschikt is voor de ontwikkeling van een kostenefficiënte en schaalbare mobiele app. Hierbij worden de voor- en nadelen van drie benaderingen onderzocht: native apps (zoals voor Android en iOS), Progressive Web Apps (PWA), en cross-platform frameworks (zoals React Native en Flutter). De hoofdonderzoeksvraag luidt: "Welke benadering is het meest geschikt voor de ontwikkeling van de mobiele app en hoe kan deze op een veilige manier user management en notificaties integreren?" De voorgestelde methodologie omvat een literatuurstudie van de verschillende technologieën, gevolgd door de ontwikkeling van een Proof of Concept (POC) om de haalbaarheid van de gekozen oplossing te testen. De verwachte resultaten zijn een gedetailleerde vergelijking van de technische mogelijkheden van de verschillende benaderingen, evenals een werkend prototype van de app. Indien de resultaten de haalbaarheid en effectiviteit van de gekozen oplossing bevestigen, zal de bevestiging zijn dat een mobiele app een aanzienlijke verbetering kan bieden in de communicatie en foutdiagnose, wat zal leiden tot hogere klanttevredenheid en een efficiënter gebruik van het systeem.
\end{abstract}

\tableofcontents

% De hoofdtekst van het voorstel zit in een apart bestand, zodat het makkelijk
% kan opgenomen worden in de bijlagen van de bachelorproef zelf.
%---------- Inleiding ---------------------------------------------------------

% TODO: Is dit voorstel gebaseerd op een paper van Research Methods die je
% vorig jaar hebt ingediend? Heb je daarbij eventueel samengewerkt met een
% andere student?
% Zo ja, haal dan de tekst hieronder uit commentaar en pas aan.

%\paragraph{Opmerking}

% Dit voorstel is gebaseerd op het onderzoeksvoorstel dat werd geschreven in het
% kader van het vak Research Methods dat ik (vorig/dit) academiejaar heb
% uitgewerkt (met medesturent VOORNAAM NAAM als mede-auteur).
% 

\section{Inleiding}%
\label{sec:inleiding}

\noindent Deze bachelorproef richt zich op de ontwikkeling van een mobiele applicatie voor SmartKit, het energiebeheersysteem van ATS. Dit systeem monitort en stuurt energiestromen zoals zonnepanelen, batterijen, laadpalen en verwarmingssystemen. Momenteel worden meldingen over systeemfouten, zoals storingen in de omvormer of een volledig opgeladen elektrisch voertuig, voornamelijk via e-mail verstuurd. Deze aanpak leidt echter tot een overvloed aan berichten, waardoor het voor beheerders moeilijk wordt om belangrijke meldingen snel op te merken en adequaat te reageren. Dit vertraagt de reactie op mogelijke systeemstoringen en verhoogt de kans op gemiste belangrijke informatie. \\

\noindent Het doel van deze bachelorproef is om te onderzoeken hoe de communicatie van deze meldingen verbeterd kan worden door over te schakelen van e-mail naar mobiele notificaties. Door gebruik te maken van een mobiele applicatie kunnen meldingen directer, sneller en doelgerichter worden overgebracht, wat de efficiëntie van het systeembeheer ten goede komt. De centrale onderzoeksvraag die in dit onderzoek wordt behandeld, is: “Welke technologie is het meest geschikt voor de ontwikkeling van een mobiele applicatie die systeemmeldingen efficiënt en veilig verstuurt, en welke technologie kan op een kostenefficiënte manier integreren met het bestaande SmartKit-systeem?” \\

\noindent Om deze vraag te beantwoorden, worden verschillende deelvragen geformuleerd die zich richten op specifieke aspecten van de oplossing. De eerste vraag onderzoekt de keuze tussen verschillende technologieën: “Wat zijn de voor- en nadelen van native apps, Progressive Web Apps (PWA) en cross-platform frameworks voor de ontwikkeling van de mobiele applicatie?” Deze vraag richt zich op het vergelijken van de mogelijkheden van deze technologieën, met nadruk op hun prestaties, gebruikservaring en ontwikkeltijd.\\

\noindent Daarnaast wordt er gekeken naar de integratie van user management: “Hoe kan de mobiele applicatie op een veilige manier gebruikers beheren en toegang reguleren?” Aangezien de app gevoelige systeemdata verwerkt, is het essentieel dat alleen geautoriseerde gebruikers toegang hebben op basis van hun rol en verantwoordelijkheden binnen het systeem.

\noindent Een andere belangrijke vraag is: “Welke technologie biedt de beste mogelijkheden voor het efficiënt versturen van mobiele notificaties binnen het SmartKit-systeem?” Hier wordt onderzocht welke systemen en technologieën het meest geschikt zijn om meldingen snel en effectief naar de juiste gebruikers te sturen.\\

\noindent Tot slot wordt er gekeken naar de kosten- en schaalvoordelen van de technologieën: “Wat zijn de kosten- en schaalvoordelen van de verschillende technologieën voor de ontwikkeling van een mobiele app binnen een klein ontwikkelteam?” Deze vraag richt zich op de praktische haalbaarheid van de gekozen technologieën, met aandacht voor de beschikbare middelen en de capaciteit van het ontwikkelteam bij ATS.

\noindent Het uiteindelijke resultaat van dit onderzoek zal een gedetailleerd rapport zijn, aangevuld met een Proof of Concept (PoC) die de gekozen oplossing in de praktijk demonstreert. Het PoC zal aantonen hoe de technologieën kunnen worden geïntegreerd om systeemmeldingen snel en efficiënt naar gebruikers te sturen. Dit zal de communicatie tussen ATS en haar klanten verbeteren, wat de klanttevredenheid verhoogt en bijdraagt aan een effectievere werking van het SmartKit-systeem.

%---------- Stand van zaken ---------------------------------------------------

\section{Literatuurstudie}% \label{sec:literatuurstudie}

\noindent In de afgelopen jaren is mobiele app-ontwikkeling een essentieel onderdeel geworden van de digitale strategieën van bedrijven wereldwijd. Met de voortdurende vooruitgang in technologieën en gebruikersbehoeften zijn er verschillende benaderingen ontstaan voor het ontwikkelen van mobiele apps. De belangrijkste benaderingen zijn native apps, hybride apps, en progressieve webapps (PWA's). Elk van deze benaderingen heeft zijn eigen voordelen en beperkingen, afhankelijk van de context van gebruik, ontwikkelingskosten, prestatie-eisen en de doelmarkt.

\subsection{Native vs. Hybride Apps} 
\noindent Een belangrijke discussie in mobiele app-ontwikkeling gaat over de keuze tussen native en hybride apps. Native apps zijn specifiek ontwikkeld voor een bepaald besturingssysteem, zoals Android of iOS, met gebruik van de native programmeertalen en tools van het platform, zoals Java/Kotlin voor Android en Swift/Objective-C voor iOS. Deze apps bieden doorgaans de beste prestaties en de meest geavanceerde toegang tot hardwarefunctionaliteiten zoals de camera, GPS, en sensoren. Ze zijn ook vaak de voorkeur bij toepassingen die hoge snelheid of intensieve grafische prestaties vereisen, zoals games of augmented reality-apps \autocite{Lau2022}.\\

\noindent Aan de andere kant bieden hybride apps de voordelen van cross-platform compatibiliteit door gebruik te maken van webtechnologieën zoals HTML, CSS en JavaScript. Hybride apps worden vaak gepromoot vanwege hun lagere ontwikkelings- en onderhoudskosten, aangezien één codebase kan worden ingezet op meerdere platforms \autocite{Singh2024}. De prestaties van hybride apps kunnen echter variëren, en ze kunnen trager zijn dan native apps, vooral als ze complexe gebruikersinterfaces of zware grafische processen bevatten.\\

\noindent De keuze tussen native en hybride apps hangt dus sterk af van de specifieke eisen van een project. Voor apps die optimale prestaties vereisen, zoals real-time toepassingen of applicaties die veel interactie met hardware nodig hebben, is de keuze voor native apps vaak gerechtvaardigd. Voor eenvoudigere apps die geen zware grafische verwerking vereisen, kunnen hybride apps een kosteneffectieve oplossing zijn \autocite{Microsoft}.

\subsection{Cross-Platform Apps} 
\noindent Naast native en hybride apps, is de opkomst van cross-platform ontwikkeltools zoals React Native, Flutter, en Xamarin ook relevant. Deze tools stellen ontwikkelaars in staat om apps te schrijven die op meerdere platformen draaien, met een enkele codebase, maar met performance die dichter in de buurt komt van native apps dan traditionele hybride apps. React Native en Flutter zijn bijvoorbeeld in staat om een groot aantal native componenten direct aan te spreken, wat de performance aanzienlijk verbetert \autocite{Soegaard2024}.

\noindent Deze cross-platform benaderingen hebben het voordeel van snellere ontwikkelingscycli en lagere kosten, vergelijkbaar met hybride apps, maar bieden betere prestaties en meer native-achtige ervaringen. Dit maakt ze een populaire keuze voor veel bedrijven die apps willen ontwikkelen die zowel op iOS als Android beschikbaar moeten zijn \autocite{Amazon}. Gezien de voorkeur voor het .NET-framework binnen dit onderzoek, komt .NET MAUI naar voren als een geschikte keuze voor cross-platform ontwikkeling \autocite{Dijk2022}, omdat het de voordelen van een enkel codebase biedt, vergelijkbaar met Xamarin, maar zonder de nadelen van het verouderde Xamarin-platform.

\subsection{Progressieve Web Apps (PWA’s)} 
\noindent Een andere benadering die steeds meer aandacht krijgt, zijn progressieve web apps (PWA's). PWA's combineren de voordelen van mobiele apps en webapplicaties door gebruik te maken van moderne webtechnologieën om een app-achtige ervaring te bieden in een browser. Ze zijn platformonafhankelijk, kunnen offline functioneren, en bieden snelle laadtijden door caching \autocite{Fortunato2018}. In vergelijking met native en hybride apps kunnen PWA's sneller worden ontwikkeld en onderhouden, aangezien ze geen aparte versies voor verschillende besturingssystemen vereisen. Bovendien kunnen ze direct via de browser worden gedistribueerd, zonder tussenkomst van app stores \autocite{BioernHansen2018}.

\noindent PWA’s hebben echter beperkingen op het gebied van hardware-integratie en gebruikersinteractie. Ze kunnen bijvoorbeeld minder diep integreren met de lokale opslag of de camera dan native apps. Desondanks zijn ze vooral aantrekkelijk voor bedrijven die snel nieuwe functies willen implementeren zonder zich zorgen te maken over app store goedkeuring of platform-specifieke eisen \autocite{Mozilla}.

\subsection{Beveiliging van Mobiele Apps} 
\noindent De beveiliging van mobiele apps is een cruciaal onderwerp in de hedendaagse app-ontwikkeling. Met de groeiende hoeveelheid persoonlijke gegevens die via mobiele apps worden gedeeld, is het essentieel om de veiligheid van apps te waarborgen. Onderzoek heeft aangetoond dat mobiele apps vaak kwetsbaar zijn voor beveiligingslekken, zoals onveilige gegevensopslag, onversleutelde communicatie, en onvoldoende bescherming tegen malware \autocite{Zhu2014}. Het is van groot belang dat ontwikkelaars beveiligingsmaatregelen zoals encryptie, veilige communicatiekanalen, en sterke authenticatie implementeren om de risico’s te minimaliseren.\\

\noindent Gezien de focus van dit onderzoek op veilige inlogmethoden, zoals tweefactorauthenticatie (2FA) via Azure AD of Google \autocite{Aussems2021}, is het belangrijk te benadrukken dat deze methoden de veiligheid van de app aanzienlijk kunnen verbeteren door een extra laag van bescherming toe te voegen bij het inloggen. Recent onderzoek heeft zich gericht op de verbetering van de beveiligingsmaatregelen in hybride en cross-platform apps, aangezien deze vaak meer kwetsbaar zijn dan native apps door de afhankelijkheid van externe frameworks en plug-ins \autocite{Wang2015}. Bovendien wordt er steeds meer gebruikgemaakt van beveiligingstechnologieën zoals containerization en runtime protection om de app-beveiliging verder te versterken \autocite{Weichbroth2020}.

\subsection{User Management en Notificaties} 
\noindent Naast de technische aspecten van app-ontwikkeling zijn er ook belangrijke overwegingen met betrekking tot gebruikersbeheer en notificaties. Het effectief beheren van gebruikersaccounts, toegangspunten en data is essentieel voor de algehele functionaliteit en veiligheid van de app. Bovendien spelen notificaties een cruciale rol in het betrekken van gebruikers en het verbeteren van de gebruikerservaring. Push-notificaties, zowel voor native als hybride apps, kunnen worden gebruikt om gebruikers op de hoogte te houden van belangrijke gebeurtenissen, promoties, of updates \autocite{Android2024}. Het gebruik van gepersonaliseerde en contextuele notificaties kan echter leiden tot een beter gebruikersengagement en klanttevredenheid \autocite{Sarin}.\\

\noindent De mogelijkheid om notificaties te koppelen aan acties zoals het 'acknowledgen' van berichten of het openen van specifieke pagina's binnen de app is van groot belang voor de interactie van gebruikers met de app. Dit biedt niet alleen verbeterde gebruikersbetrokkenheid, maar kan ook de efficiëntie en functionaliteit van de app verhogen, vooral in de context van bedrijfsomgevingen zoals het SmartKit-systeem van ATS.



%---------- Methodologie ------------------------------------------------------
\section{Methodologie}
\label{sec:methodologie}

\noindent Dit onderzoek richt zich op de ontwikkeling van een mobiele applicatie voor het SmartKit-energiebeheersysteem van ATS, met een focus op veilige inlogmethoden (zoals 2FA via Azure AD of Google) en notificatiesystemen waarmee gebruikers meldingen kunnen ontvangen en acties kunnen ondernemen (zoals het 'acknowledgen' van meldingen of het openen van specifieke pagina's binnen de app). Hierbij wordt specifiek gekeken naar welke meldingen er gestuurd moeten worden naar de gebruikers van het SmartKit-systeem en hoe deze meldingen effectief beheerd kunnen worden. Het doel is de efficiëntie van systeembeheer te verbeteren door meldingen sneller en veiliger over te brengen dan het huidige e-mailsysteem. \\

\noindent Het gekozen framework voor de ontwikkeling van de applicatie zal een belangrijke overweging zijn, maar wordt pas na een grondige evaluatie van verschillende opties definitief gemaakt. In dit onderzoek wordt .NET MAUI zeker meegenomen in de evaluatie vanwege de integratie met het .NET-ecosysteem, de voordelen van cross-platform ontwikkeling en de ondersteuning voor moderne technologieën zoals notificatiesystemen en veilige inlogmethoden. Andere frameworks, zoals React Native en Flutter, zullen eveneens worden onderzocht om te bepalen welke het beste aansluit bij de eisen van het SmartKit-systeem en de doelstellingen van het onderzoek.

\subsection{Literatuurstudie en Technologische \\Verkenning (1-2 weken)}
\noindent In de eerste fase wordt een literatuurstudie uitgevoerd om inzicht te krijgen in de verschillende technologieën die mogelijk geschikt zijn voor de ontwikkeling van de mobiele applicatie. Deze studie richt zich op de voor- en nadelen van native apps, Progressive Web Apps (PWA), en cross-platform frameworks. Daarnaast worden relevante publicaties over mobiele notificatiesystemen, gebruikersbeheer, en beveiligingstechnieken zoals 2FA en encryptie geanalyseerd, zodat de beste technologie kan worden geselecteerd voor het SmartKit-systeem. Hierbij wordt onderzocht welke technologieën het beste kunnen ondersteunen in het versturen van meldingen naar gebruikers en hoe deze meldingen veilig beheerd kunnen worden. \\

\noindent \textbf{Deliverable:} Rapport met een overzicht van de verschillende technologieën, hun voor- en nadelen, en een keuze voor de meest geschikte technologie, met bijzondere aandacht voor beveiliging en notificatiesystemen.

\subsection{Requirements-analyse (2-3 weken)}
\noindent De tweede fase bestaat uit het uitvoeren van interviews met verschillende belanghebbenden, zoals systeembeheerders, IT-specialisten en eindgebruikers van het SmartKit-systeem. Deze interviews helpen bij het verzamelen van gedetailleerde functionele en niet-functionele eisen voor de mobiele applicatie. De vragen richten zich onder andere op welke meldingen belangrijk zijn om te versturen, hoe gebruikersbeheer moet worden ingericht, en welke integratie met het bestaande SmartKit-systeem noodzakelijk is. Tevens worden de eisen voor veilige inlogmethoden (zoals 2FA) besproken. Het onderzoek zal hierbij focussen op welke meldingen verstuurd moeten worden en hoe gebruikers deze kunnen ontvangen en beheren. \\

\noindent \textbf{Deliverable:} Gedocumenteerde vereisten voor de mobiele applicatie, inclusief technische en functionele specificaties op basis van de interviews, met focus op beveiliging en notificatiefunctionaliteiten.

\subsection{Vergelijkende Studie van \\Technologieën (1-2 weken)}
\noindent Op basis van de literatuurstudie en de interviews wordt een vergelijkende studie uitgevoerd om de drie geselecteerde technologieën (native apps, PWA, en cross-platform frameworks) te evalueren. Hierbij worden de prestaties, ontwikkeltijd, kosten, gebruikservaring, schaalbaarheid en veiligheid van elke technologie vergeleken. Deze fase helpt om de meest geschikte technologie te kiezen, rekening houdend met de behoeften van ATS en de beschikbare middelen voor het ontwikkelteam. Bij de beoordeling van cross-platform frameworks zal specifiek .NET MAUI worden vergeleken met andere tools zoals React Native en Flutter, waarbij onder andere gekeken wordt naar de integratie van notificatiesystemen en de mogelijkheid om meldingen efficiënt te versturen. \\

\noindent \textbf{Deliverable:} Een gedetailleerde analyse van de drie technologieën, inclusief aanbevelingen voor de keuze van de technologie, met nadruk op beveiliging en integratie van notificatiesystemen.

\subsection{Ontwikkeling van de Proof of \\Concept (PoC) (4-5 weken)}
\noindent In de vierde fase wordt een Proof of Concept (PoC) ontwikkeld. De PoC is een werkend prototype van de mobiele applicatie, die minimaal de kernfunctionaliteit van meldingen via mobiele notificaties bevat. Het PoC wordt ontwikkeld in .NET MAUI en integreert met het SmartKit-systeem. De focus ligt op de testbaarheid van de belangrijkste functies, zoals het ontvangen van notificaties, de interactie met de backend van SmartKit en de implementatie van veilige inlogmethoden zoals 2FA. Hierbij wordt specifiek onderzocht hoe de notificatiesysteemfunctionaliteit geoptimaliseerd kan worden, zodat meldingen tijdig en veilig naar gebruikers kunnen worden gestuurd. \\

\noindent \textbf{Deliverable:} Werkend prototype van de mobiele applicatie, met de basisfunctionaliteit voor notificaties, veilige inlogmethoden, en integratie met SmartKit.

\subsection{Testen van de PoC en Verzamelen \\van Feedback (2 weken)}
\noindent Na de ontwikkeling van de PoC wordt de applicatie getest door een selecte groep gebruikers, waaronder systeembeheerders en IT-specialisten van ATS. Het doel is om de werking van de notificaties, de gebruikersinterface, de beveiliging van het inlogsysteem en de integratie met het SmartKit-systeem te evalueren. Feedback wordt verzameld en geanalyseerd om de gebruikservaring te verbeteren en eventuele technische problemen te identificeren. Dit proces maakt het mogelijk om de app verder te optimaliseren en de meldingen te verfijnen op basis van de voorkeuren van de gebruikers. \\

\noindent \textbf{Deliverable:} Testresultaten en verzamelde feedback, inclusief verbeterpunten voor de app, met nadruk op de beveiliging en gebruikservaring van notificaties.

\subsection{Risico-analyse en Beveiligings\overwegingen (1 week)}
\noindent In deze fase wordt een risico-analyse uitgevoerd om potentiële risico’s en technische uitdagingen te identificeren, met bijzondere aandacht voor beveiliging en prestaties. Aangezien de mobiele app gevoelige systeemdata verwerkt, is het belangrijk om te zorgen voor veilige communicatie, gebruikersauthenticatie en integriteit van notificaties. De risico-analyse zal ook mogelijke knelpunten bij de integratie met SmartKit in kaart brengen en strategieën voorstellen voor het mitigeren van deze risico’s, met nadruk op het beveiligen van notificaties en de gegevensoverdracht. \\

\noindent \textbf{Deliverable:} Risico-analyserapport met aanbevelingen voor het verminderen van technische en beveiligingsrisico’s, inclusief de implementatie van veilige inlogmethoden en notificatiesystemen.

\subsection{Eindrapport en Conclusies \\(2 weken)}
\noindent De laatste fase van het onderzoek bestaat uit het schrijven van het eindrapport, waarin alle bevindingen, technische keuzes, testresultaten, en risico-analyse worden samengevat. Het rapport bevat een gedetailleerde beschrijving van het ontwikkelde PoC en de keuze van de technologieën, evenals een evaluatie van de effectiviteit van de oplossing. Het eindrapport bevat ook aanbevelingen voor de verdere implementatie van de mobiele applicatie binnen ATS, inclusief verbeteringen op het gebied van notificaties en de bijbehorende gebruikersinteractie. \\

\noindent \textbf{Deliverable:} Eindrapport met een samenvatting van het onderzoek, technische keuzes, testresultaten en aanbevelingen voor de implementatie.



%---------- Verwachte resultaten ----------------------------------------------
\section{Verwacht Resultaat en Conclusies}
\label{sec:verwachte_resultaten}

\noindent In dit gedeelte worden de verwachte resultaten van het onderzoek samengevat, gevolgd door de conclusies die uit deze resultaten getrokken kunnen worden.

\subsection{Keuze van de Technologie}
\noindent Het onderzoek zal naar verwachting uitmonden in de keuze voor .NET MAUI als cross-platform oplossing. Het biedt sterke integratie met het .NET-ecosysteem, wat het een uitstekende keuze maakt voor de ontwikkeling van de mobiele applicatie binnen dit onderzoek. Het ondersteunt de ontwikkeling van apps voor zowel iOS als Android, en stelt ons in staat om gebruik te maken van C#, wat goed aansluit bij de bestaande technologieën binnen het SmartKit-systeem. Andere technologieën, zoals React Native en Flutter, worden in dit geval niet verder onderzocht, aangezien .NET MAUI de beste integratie en ondersteuning biedt voor de specifieke behoeften van het project. Native apps zouden theoretisch betere prestaties kunnen leveren, maar de extra ontwikkeltijd en kosten maken dit minder praktisch voor een klein team. PWA's worden niet overwogen, aangezien ze onvoldoende functionaliteit bieden voor de diepgaande systeemintegratie die vereist is voor het SmartKit-systeem.\\


\noindent \textbf{Conclusie:} Cross-platform frameworks bieden de beste oplossing voor de benodigde snelheid, kostenbesparing en schaalbaarheid voor dit project.

\subsection{Verbeterde Communicatie via \\Push-notificaties}
\noindent Door de verschuiving van e-mailmeldingen naar \textbf{push-notificaties} binnen de mobiele applicatie wordt verwacht dat de snelheid en effectiviteit van de meldingen aanzienlijk zal verbeteren. Dit zal de responstijd op systeemstoringen verkorten, aangezien meldingen directer en meer gericht naar de juiste gebruikers worden gestuurd.\\


\noindent \textbf{Conclusie:} De integratie van push-notificaties zal de efficiëntie van het systeembeheer verbeteren en de klanttevredenheid verhogen door snel en tijdig reageren op storingen.

\subsection{Beveiliging en Gebruikersbeheer}
\noindent Aangezien de app toegang biedt tot gevoelige systeeminformatie, zal er sterke aandacht zijn voor \textbf{gebruikersbeheer} en \textbf{beveiliging}. Het wordt verwacht dat technologieën zoals OAuth2 en JWT worden geïmplementeerd voor veilige authenticatie en autorisatie van gebruikers. Dit garandeert dat alleen geautoriseerde gebruikers toegang hebben tot systeemdata.\\


\noindent \textbf{Conclusie:} Een robuust gebruikersbeheer en beveiligingsprotocol is essentieel voor het waarborgen van de integriteit en vertrouwelijkheid van het systeem, en zal ervoor zorgen dat enkel bevoegde personen toegang hebben.

\subsection{Proof of Concept (PoC) en \\Testresultaten}
\noindent De \textbf{Proof of Concept (PoC)} zal een werkend prototype van de applicatie opleveren, met minimaal de basisfunctionaliteit voor meldingen en integratie met het SmartKit-systeem. Verwacht wordt dat de PoC effectief zal demonstreren dat het systeem in staat is om push-notificaties te verzenden en snel te reageren op systeemfouten. De testresultaten zullen ook waardevolle feedback opleveren voor de verdere verfijning van de app. \\


\noindent \textbf{Conclusie:} De PoC zal aantonen dat de geselecteerde technologie en notificatiesysteem daadwerkelijk effectief functioneren, en de feedback zal helpen bij het verder optimaliseren van de gebruikerservaring en technische prestaties.

\subsection{Praktische Haalbaarheid en \\Verdere Implementatie}
\noindent Het onderzoek verwacht dat de mobiele applicatie \textbf{technisch haalbaar} is en goed kan worden geïmplementeerd binnen het beschikbare tijd- en budgetkader van ATS. De keuze voor een cross-platform oplossing maakt het mogelijk om het project met beperkte middelen effectief uit te voeren. Het PoC zal dienen als fundament voor verdere implementatie en uitrol van de app in productie. \\


\noindent \textbf{Conclusie:} De mobiele applicatie is een haalbare en kosteneffectieve oplossing voor ATS, met een goed potentieel voor verdere uitbreiding en implementatie, gebaseerd op de keuze van technologieën en de feedback uit de testfase.




\printbibliography[heading=bibintoc]

\end{document}